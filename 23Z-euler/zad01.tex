\textbf{\large\color{orange}Zadanie 1.} Opisz grupę automorfizmów triangulacji $\R P^2$ o najmniejszej liczbie wierzchołków.

%{\large\color{red}usunąć kolokwializmy, pokazać, że jądro $Aut(D)\to S_5=\Z_2$, na początku uzasadnić $2E=3T$, ładniej pokazać, że sześciany są trzymane przez automorfizmy}

\textbf{Ile wierzchołków?}

Zacznijmy od zauważenia, że patrzymy na triangulację na $2$-rozmaitości. To znaczy, że dowolny punkt triangulacji ma otoczenie, które wygląda jak mała sfera w $\R^2$. Na $\R^2$ każda krawędź przylega do $2$ trójkątów, a każdy trójkąt ma $3$ krawędzie, stąd
$$2E=3T.$$

Wiemy, że jeśli $X$ ma triangulację o $V$ wierzchołkach, $E$ krawędziach i $T$ trójkątach, to 
$$\chi(X)=V-E+T.$$
Jak wcześniej zauważyliśmy, dla $\R P^2$ $2E=3T$, więc mamy
$$\chi(\R P^2)=V-E+\frac{2}{3}E=V-\frac{1}{3}E.$$
Ilość krawędzi szacujemy od góry przez ilość krawędzi w grafie pełnym: $E\leq \binom{V}{2}$ czyli
$$V=\chi(\R P^2)+\frac{1}{3}E\leq \chi(\R P^2)+\frac{V(V-1)}{6}$$
dla $\R P^2$ $\chi(\R P^2)=1$, dostajemy więc ograniczenie
$$V\leq 1+\frac{V(V-1)}{6}$$
$$6\geq 6V-V^2+V=V(7-V)$$
Powyższa nierówność dla $V=6$ staje się równością. Tak samo dla $V=1$ mamy równość, ale z oczywistego powodu nie ma jednowierzchołkowej triangulacji na $\R P^2$. Pozostałe liczby naturalne z przedziału $(0, 7)$ nie mają szansy spełniać powyższe równanie (widać na obrazku)
\begin{center}
  \begin{tikzpicture}
  \begin{axis}[
    axis x line=center,
    axis y line=center,
    xtick={0, 2, 4, 6, 8},
    ytick={0, 2, 4, 6, 8, 10, 12},
    xlabel={$x$},
    ylabel={$y$},
    xlabel style={below right},
    ylabel style={left},
    xmin=-1.5,
    xmax=8.5,
    ymin=-1.5,
    ymax=13.5]
  \addplot [mark=none,domain=-1:7] {x*(7-x)};
  \addplot [domain=-1:7, dashed] {6};
  \addplot [dashed] coordinates {(6, -0.5) (6, 12)};
  \end{axis}
  \end{tikzpicture}
\end{center}

Z listy 1 wiemy, że $6$ wierzchołkowa triangulacja $\R P^2$ jest jedyna z dokładnością do izomorfizmu, czyli nie musimy się martwić którą triangulacje opisujemy.

\textbf{Plan działania}

Płaszczyzna rzutowa $\R P^2$ to $S^2$ wydzielona przez antypodyczne działanie $\Z_2$. W takim razie, $6$ wierzchołkowa triangulacja na $\R P^2$ przychodzi od triangulacji $12$ wierzchołkowej na $S^2$. Dwudziestościan ma $12$ wierzchołków i $20$ ścian i jest to interesująca nas triangulacja sfery. Łatwiejsze jest jednak badanie grupy automorfizmów bryły dualnej do dwudziestościanu - dwunastościanu o $12$ ścianach i $20$ wierzchołkach.

%{\large\color{red}narysować na sferze z osiami symetrii}

\tdplotsetmaincoords{70}{110}
Ponieważ dwudziestościan i dwunastościan są bryłami dualnymi, to grupy ich automorfizmów są sobie równe. Grupa automorfizmów jest generowana przez odwzorowanie $x\mapsto -x$ i symetrie:

\begin{wrapfigure}{r}{6cm}
\begin{center}
\begin{tikzpicture}
  \begin{scope}[tdplot_main_coords]
	% Change the value of the number at {\escala}{##} to scale the figure up or down
	\pgfmathsetmacro{\escala}{2}
	% Coordinates of the vertices
%  \node(1) at (-\escala*1.37638, \escala*0., \escala*0.262866) {1};
%  \node(2) at (\escala*1.37638, \escala*0., -\escala*0.262866) {2};
%  \node(3) at (-\escala*0.425325, -\escala*1.30902, \escala*0.262866) {3};
%  \node(4) at (-\escala*0.425325, \escala*1.30902, \escala*0.262866) {4};
%  \node(5) at (\escala*1.11352, -\escala*0.809017, \escala*0.262866) {5};
%  \node(6) at (\escala*1.11352, \escala*0.809017, \escala*0.262866) {6};
%  \node(7) at (-\escala*0.262866, -\escala*0.809017, \escala*1.11352) {7};
%  \node(8) at (-\escala*0.262866, \escala*0.809017, \escala*1.11352) {8};
%  \node(9) at (-\escala*0.688191, -\escala*0.5, -\escala*1.11352) {9};
%  \node(10) at (-\escala*0.688191, \escala*0.5, -\escala*1.11352) {10};
%%	%
%  \node(11) at (\escala*0.688191, -\escala*0.5, \escala*1.11352) {11};
%  \node(12) at (\escala*0.688191, \escala*0.5, \escala*1.11352) {12};
%  \node(13) at (\escala*0.850651, \escala*0., -\escala*1.11352) {13};
%  \node(14) at (-\escala*1.11352, -\escala*0.809017, -\escala*0.262866) {14};
%  \node(15) at (-\escala*1.11352, \escala*0.809017, -\escala*0.262866) {15};
%  \node(16) at (-\escala*0.850651, \escala*0., \escala*1.11352) {16};
%  \node(17) at (\escala*0.262866, -\escala*0.809017, -\escala*1.11352) {17};
%  \node(18) at (\escala*0.262866, \escala*0.809017, -\escala*1.11352) {18};
%  \node(19) at (\escala*0.425325, -\escala*1.30902, -\escala*0.262866) {19};
%  \node(20) at (\escala*0.425325, \escala*1.30902, -\escala*0.262866) {20};
	% Faces of the dodecahedron

	\coordinate(1) at (-\escala*1.37638, \escala*0., \escala*0.262866);
	\coordinate(2) at (\escala*1.37638, \escala*0., -\escala*0.262866);
	\coordinate(3) at (-\escala*0.425325, -\escala*1.30902, \escala*0.262866);
	\coordinate(4) at (-\escala*0.425325, \escala*1.30902, \escala*0.262866);
	\coordinate(5) at (\escala*1.11352, -\escala*0.809017, \escala*0.262866);
	\coordinate(6) at (\escala*1.11352, \escala*0.809017, \escala*0.262866);
	\coordinate(7) at (-\escala*0.262866, -\escala*0.809017, \escala*1.11352);
	\coordinate(8) at (-\escala*0.262866, \escala*0.809017, \escala*1.11352);
	\coordinate(9) at (-\escala*0.688191, -\escala*0.5, -\escala*1.11352);
	\coordinate(10) at (-\escala*0.688191, \escala*0.5, -\escala*1.11352);
	%
	\coordinate(11) at (\escala*0.688191, -\escala*0.5, \escala*1.11352);
	\coordinate(12) at (\escala*0.688191, \escala*0.5, \escala*1.11352);
	\coordinate(13) at (\escala*0.850651, \escala*0., -\escala*1.11352);
	\coordinate(14) at (-\escala*1.11352, -\escala*0.809017, -\escala*0.262866);
	\coordinate(15) at (-\escala*1.11352, \escala*0.809017, -\escala*0.262866);
	\coordinate(16) at (-\escala*0.850651, \escala*0., \escala*1.11352);
	\coordinate(17) at (\escala*0.262866, -\escala*0.809017, -\escala*1.11352);
	\coordinate(18) at (\escala*0.262866, \escala*0.809017, -\escala*1.11352);
	\coordinate(19) at (\escala*0.425325, -\escala*1.30902, -\escala*0.262866);
	\coordinate(20) at (\escala*0.425325, \escala*1.30902, -\escala*0.262866);

  \draw[dashed, thick] (20)--(6)--(2)--(13);
  \draw[dashed, thick] (6)--(12)--(11)--(5)--(2)--(6);
  \draw[dashed, thick] (12)--(8);
  \draw[dashed, thick] (7)--(11);
  \draw[dashed, thick] (5)--(19);
  \draw[thick] (7)--(16)--(1)-- coordinate (114) (14)--(3)--(7);
  \draw[thick] (16)--(8)--(4)--(15)-- coordinate (151) (1);
  \draw[thick] (14)-- coordinate (149) (9)-- coordinate (910) (10)-- coordinate (1015) (15);
  \draw[thick] (3)--(19)--(17)--(9);
  \draw[thick] (17)--(13)--(18)--(10);
  \draw[thick] (18)--(20)--(4);
\fill [opacity=0.8, white] (-7, -7) rectangle (7, 7);

\draw[very thick, red] (14)--(1015);
\draw[very thick, red] (10)--(114);
\draw[very thick, red] (151)--(9);
\draw[very thick, red] (149)--(15);
\draw[very thick, red] (910)--(1);

\draw[very thick, blue] (14)--(9)--(10)--(15)--(1)--(14); 

\end{scope}

\draw[thick] (0,0) circle (2.8);
\draw[thick] (-2.8, 0) arc (180:360:2.8 and 1.4);
\draw[thick, dashed] (-2.8, 0) arc (180:0:2.8 and 1.4);


\end{tikzpicture}
\end{center}
\end{wrapfigure}

odbicia dwunastościanu względem płaszczyzn przecinających wybraną ścianę (przekrój ściany z płaszczyzną zaznaczono na czerwono) odpowiadają odbiciom dwudziestościanu względem płaszczyzn przecinających odpowiadającą krawędź tej bryły. Natomiast odbicia względem płaszczyzn przechodzących przez krawędzie dwunastościanu odpowiadają odbiciom względem prostych przecinających odpowiadającą tej krawędzi ścianę dwudziestościanu.

%czerwone przekątne ściany dwunastościanu, które odpowiadają odbiciom względem prostych przez nie przechodzących, odpowiadają w dwudziestościanie odbiciom względem 


%czerwone symetrie w dwunastościanie nakrywają się na krawędzie w dwudziestościanie, które również dają symetrie. Tak samo różowe krawędzie nakrywają się z wysokościami ścian w dwudziestościanie, które również dają osie symetrii. Czyli dwudziestościan i dwunastościan dzielą ze sobą osie symetrii.
Oznaczmy dwunastościan przez $D$. Pokażemy, że $Aut(D)/\Z_2=A_5$.

\textbf{Czy zgadza się rząd?}

Niech $v\in D$ będzie wierzchołkiem dwunastościanu (odpowiada ścianie dwudziestościanu). 
\begin{itemize}
  \item $|Obr(v)|=20$, bo automorfizm może posłać wierzchołek na dowolny inny spośród $20$ które $D$ posiada.
  \item $|Stab(v)| = 3!=6$, gdyż są to permutacje $3$ sąsiadów tego wierzchołka przy trzymaniu $v$ w miejscu.
\end{itemize}
W takim razie dostajemy
$$|Aut(D)|=|Orb(v)|\cdot|Stab(v)|=20\cdot 6=120=|A_5\times \Z_2|.$$

\newpage

\begin{wrapfigure}{l}{6cm}
\begin{tikzpicture}[tdplot_main_coords]
	% Change the value of the number at {\escala}{##} to scale the figure up or down
	\pgfmathsetmacro{\escala}{2}
	% Coordinates of the vertices
%  \node(1) at (-\escala*1.37638, \escala*0., \escala*0.262866) {1};
%  \node(2) at (\escala*1.37638, \escala*0., -\escala*0.262866) {2};
%  \node(3) at (-\escala*0.425325, -\escala*1.30902, \escala*0.262866) {3};
%  \node(4) at (-\escala*0.425325, \escala*1.30902, \escala*0.262866) {4};
%  \node(5) at (\escala*1.11352, -\escala*0.809017, \escala*0.262866) {5};
%  \node(6) at (\escala*1.11352, \escala*0.809017, \escala*0.262866) {6};
%  \node(7) at (-\escala*0.262866, -\escala*0.809017, \escala*1.11352) {7};
%  \node(8) at (-\escala*0.262866, \escala*0.809017, \escala*1.11352) {8};
%  \node(9) at (-\escala*0.688191, -\escala*0.5, -\escala*1.11352) {9};
%  \node(10) at (-\escala*0.688191, \escala*0.5, -\escala*1.11352) {10};
%	%
%  \node(11) at (\escala*0.688191, -\escala*0.5, \escala*1.11352) {11};
%  \node(12) at (\escala*0.688191, \escala*0.5, \escala*1.11352) {12};
%  \node(13) at (\escala*0.850651, \escala*0., -\escala*1.11352) {13};
%  \node(14) at (-\escala*1.11352, -\escala*0.809017, -\escala*0.262866) {14};
%  \node(15) at (-\escala*1.11352, \escala*0.809017, -\escala*0.262866) {15};
%  \node(16) at (-\escala*0.850651, \escala*0., \escala*1.11352) {16};
%  \node(17) at (\escala*0.262866, -\escala*0.809017, -\escala*1.11352) {17};
%  \node(18) at (\escala*0.262866, \escala*0.809017, -\escala*1.11352) {18};
%  \node(19) at (\escala*0.425325, -\escala*1.30902, -\escala*0.262866) {19};
%  \node(20) at (\escala*0.425325, \escala*1.30902, -\escala*0.262866) {20};
	% Faces of the dodecahedron
 
	\coordinate(1) at (-\escala*1.37638, \escala*0., \escala*0.262866);
	\coordinate(2) at (\escala*1.37638, \escala*0., -\escala*0.262866);
	\coordinate(3) at (-\escala*0.425325, -\escala*1.30902, \escala*0.262866);
	\coordinate(4) at (-\escala*0.425325, \escala*1.30902, \escala*0.262866);
	\coordinate(5) at (\escala*1.11352, -\escala*0.809017, \escala*0.262866);
	\coordinate(6) at (\escala*1.11352, \escala*0.809017, \escala*0.262866);
	\coordinate(7) at (-\escala*0.262866, -\escala*0.809017, \escala*1.11352);
	\coordinate(8) at (-\escala*0.262866, \escala*0.809017, \escala*1.11352);
	\coordinate(9) at (-\escala*0.688191, -\escala*0.5, -\escala*1.11352);
	\coordinate(10) at (-\escala*0.688191, \escala*0.5, -\escala*1.11352);
	%
	\coordinate(11) at (\escala*0.688191, -\escala*0.5, \escala*1.11352);
	\coordinate(12) at (\escala*0.688191, \escala*0.5, \escala*1.11352);
	\coordinate(13) at (\escala*0.850651, \escala*0., -\escala*1.11352);
	\coordinate(14) at (-\escala*1.11352, -\escala*0.809017, -\escala*0.262866);
	\coordinate(15) at (-\escala*1.11352, \escala*0.809017, -\escala*0.262866);
	\coordinate(16) at (-\escala*0.850651, \escala*0., \escala*1.11352);
	\coordinate(17) at (\escala*0.262866, -\escala*0.809017, -\escala*1.11352);
	\coordinate(18) at (\escala*0.262866, \escala*0.809017, -\escala*1.11352);
	\coordinate(19) at (\escala*0.425325, -\escala*1.30902, -\escala*0.262866);
	\coordinate(20) at (\escala*0.425325, \escala*1.30902, -\escala*0.262866);

%	\draw[thick,cyan, opacity=0.75]  (2) -- (6) -- (12) -- (11) -- (5) -- (2);
%	\draw[thick,cyan, opacity=0.75]  (5) -- (11) -- (7) -- (3) -- (19) -- (5);
%	\draw[thick,cyan, opacity=0.75]  (11) -- (12) -- (8) -- (16) -- (7) -- (11);
%	\draw[thick,cyan, opacity=0.75]  (12) -- (6) -- (20) -- (4) -- (8) -- (12);
%	%
%	\draw[thick,cyan, opacity=0.75]  (6) -- (2) -- (13) -- (18) -- (20) -- (6);
%	\draw[thick,cyan, opacity=0.75]  (2) -- (5) -- (19) -- (17) -- (13) -- (2);
%	\draw[thick,cyan, opacity=0.75]  (4) -- (20) -- (18) -- (10) -- (15) -- (4);
%	\draw[thick,cyan, opacity=0.75]  (18) -- (13) -- (17) -- (9) -- (10) -- (18);
%	\draw[thick,cyan, opacity=0.75]  (17) -- (19) -- (3) -- (14) -- (9) -- (17);
%	%
%	\draw[thick,cyan, opacity=0.75]  (3) -- (7) -- (16) -- (1) -- (14) -- (3);
%	\draw[thick,cyan, opacity=0.75]  (16) -- (8) -- (4) -- (15) -- (1) -- (16);
%	\draw[thick,cyan, opacity=0.75]  (15) -- (10) -- (9) -- (14) -- (1) -- (15);
	%	

  \draw[orange, thick] (19)--(7)--(1)--(9)--(19);
  \draw[orange, thick, dashed] (18)--(2)--(12)--(4)--(18);
  \draw[orange, thick] (9)--(18)--(4)--(1);
  \draw[dashed, orange, thick] (7)--(12);
  \draw[dashed, orange, thick] (19)--(2);

  \draw[dashed, thick] (20)--(6)--(2)--(13);
  \draw[dashed, thick] (6)--(12)--(11)--(5)--(2)--(6);
  \draw[dashed, thick] (12)--(8);
  \draw[dashed, thick] (7)--(11);
  \draw[dashed, thick] (5)--(19);
  \draw[thick] (7)--(16)--(1)--(14)--(3)--(7);
  \draw[thick] (16)--(8)--(4)--(15)--(1);
  \draw[thick] (14)--(9)--(10)--(15);
  \draw[thick] (3)--(19)--(17)--(9);
  \draw[thick] (17)--(13)--(18)--(10);
  \draw[thick] (18)--(20)--(4);
\end{tikzpicture}
\end{wrapfigure}

\textbf{Wpisywanie sześcianów}

Zauważmy, że w każdy dwunastościan możemy włożyć $5$ kostek. Każda kostka powstaje przez:
\begin{itemize}
  \item wybranie krawędzi
  \item połączenie wierzchołków sąsiadujących z tę krawędzią w kwadrat
  \item wzięcie krawędzi po przeciwnej stronie dwudziestościanu
  \item połączenie wierzchołków tej krawędzi w kolejny kwadrat
  \item połączenie obu kwadratów w sześcian.
\end{itemize}
Co więcej, każdy sześcian możemy ustawić tak, aby jedna jego krawędź była przekątną wybranej przez nas ściany dwunastościanu. Ponumerujmy teraz te sześciany wybierając jeden i idąc dalej po przekątnych ściany zgodnie z ruchem wskazówek zegara.

\textbf{Odwzorowanie $\boldsymbol{Aut(D)\to S_5}$}

% Obrót względem prostej, która przechodzi przez przeciwległe wierzchołki sześcianu jest też ob

Prosta, która przechodzi przez przeciwległe wierzchołki sześcianu jest też prostą, która przechodzi przez przeciwległe wierzchołki dwunastościanu. Obroty względem tej prostej o kąty $120$, $240$ i $360$ stopni są więc symetriami zarówno wpisanych sześcianów jak i dwunastościanu.

Co więcej, wspomniane obroty zawsze trzymają sześciany będące na przekątnej wychodzącej z przebitego wierzchołka, a pozostałe $3$ sześciany będą permutowane ze sobą. To pozwala nam napisać odwzorowanie 
$$\phi:Aut(D)\to S_5$$
które symetrii przypisuje odpowiadającą jej permutację $3$ sześcianów. Podgrupą $S_5$ generowaną przez wszystkie $3$-cykle jest podgrupa $S_5$ mająca $5\cdot 4\cdot 3=60$ elementów, która jest jedyna i wynosi $A_5$. Stąd 
$$\img(\phi)=A_5.$$

Pozostaje nam sprawdzić które odwzorowania tworzą jądro tego odwzorowania. Aby odpowiedzieć na to pytanie, trzeba sprawdzić, jaki automorfizm zachowuje wszystkie sześciany. Oczywistym kandydatem jest identyczność, a drugim jest symetria względem środka dwudziestościanu. Pozostałe automorfizmy muszą obrócić dwudziestościan wzdłuż którejś osi symetrii przynajmniej raz, czyli indukują permutację sześcianów. Stąd, 
$$\ker(\phi)=\{id, \text{antypodyzm}\}\cong \Z_2.$$

Korzystając z twierdzenia o izomorfizmie grup, dostajemy
$$Aut(D)/\ker(\phi)\cong Aut(D)/\Z_2\cong A_5$$

W takim razie, $Aut(\triangle \R P^2)\cong Aut(D)/\Z_2\cong A_5$ tak jak chcieliśmy.







%\bigskip
%
%\textbf{Pozbycie się $\Z_2$}
%
%Wśród automorfizmów dodecahedronu $D$ mamy dwa "rodzaje" odwzorowań
%\begin{itemize}
%  \item rotacje i symetrie, które zachowują ruch wskazówek zegara przy numerowaniu sąsiadów dowolnego wierzchołka,
%  \item odwzorowanie antypodyczne tudzież symetria względem punktu w samym środku $D$, która przewraca tę kolejność do góry nogami.
%\end{itemize}
%Ten drugi rodzaj odwzorowania będzie odpowiadać za czynnik $\Z_2$ w $Aut(D)$. Wystarczy więc zająć się samą grupą symetrii i rotacji i pokazać, że to $A_5$.
%
%\textbf{Symetrie i obroty}
%
%Sztuczką na pokazanie, że symetrie $D$ to $A_5$ jest zauważenie $5$ sześcianów w środku $D$. Sześciany możemy narysować idąc krokami:
%\begin{itemize}
%  \item weź krawędź w $D$
%  \item połącz wszystkie sąsiady tej krawędzi w ścianę
%  \item weź krawędź po przeciwnej stronie $D$
%  \item połącz jej wszystkie sąsiady w ścianę
%  \item połącz te dwie ściany w sześcian.
%\end{itemize}
%Z tej metody wytwarzania sześcianów można od razu wywnioskować, że automorfizm przeprowadza sześciany na sześciany, ponieważ sąsiedztwo wierzchołków musi być zachowane, a to ono było podstawą wyciskania sześcianów z $D$.
%
%Ponumerujmy sześciany od $1$ do $5$ - możemy teraz je permutować. Najbardziej leniwym sposobem na zauważenie, że grupa uzyskana przez porządne permutacje tych sześcianów to $A_5$ jest podzielenie $|Aut(D)|=120$ przez $2$, które oznacza, że wyrzucamy antypodyzm (element rzędu $2$). Zostawia to nam $60$ automorfizmów, które będą permutować te sześciany i które powinniśmy móc włożyć w $S_5$. Jedyna (z dokładnością do izomorfizmu) podgrupa $S_5$ o $60$ elementach jest $A_5$ tak jak chcieliśmy.
%
%Uzasadniliśmy, że $A_5\times\Z_2=Aut(\text{\emph{dodecahedron}})=Aut(\text{\emph{icosahedron}})$ bo tak jak już wspomniałam, bryły te są dualne. Po wydzieleniu $S^2$ z triangulacją będącą icosahedronem przez działanie antypodyczne dostajemy grupę automorfizmów triangulacji $\Delta \R P^2$ o $6$ wierzchołkach: 
%$$Aut(\Delta\R P^2)=A_5\times\Z_2/\Z_2=A_5$$
