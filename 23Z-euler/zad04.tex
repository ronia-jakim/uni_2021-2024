\textbf{\large\color{orange}Zadanie 4.} Niech $X$ będzie przestrzenią topologiczną. Definiujemy przestrzeń konfiguracji $\Conf_n(X)$ jako przestrzeń położeń $n$ różnych punktów w $X$ (zakładamy, że dwa punkty nie mogą leżeć w tym samym miejscu. Definiujemy też przestrzeń konfiguracji $\conf_n(X)$ jako przestrzeń położeń $n$ nierozróżnialnych punktów w $X$, czyli
$$\Conf_n(X)=\{(x_1,...,x_n)\in X^n\;:\;x_i\neq x_j, \text{ gdy }i\neq j\}$$
$$\conf_n(X)=\frac{\Conf_n(X)}{S_n}$$
gdzie $S_n$ działa na $\Conf_n(X)$ przez permutacje współrzędnych.

Oblicz charaketrystykę Eulera $\conf_n(\mathbf{Y})$, gdzie $\mathbf{Y}$ to drzewo o $4$ wierzchołkach, z czego $3$ to liście, dla $n=2,3,4$.

\dotfill

\textbf{$\mathbf{n=2}$}

%Zauważmy, że w każdym punkcie $\Conf_2(\mathbf{Y})$ leży niemalże identyczna kopia $\mathbf{Y}$ z tym, że brakuje w niej jednego punktu -> tego, który miałby obie współrzędne równe. Korzystając z addytywnej definicji charakterystyki Eulera, graf $\mathbf{Y}$ ma charakterystykę $\chi(\mathbf{Y})=1$. W takim razie, jeśli wyjmiemy z niego punkt to dostajemy graf z $\chi(\mathbf{Y}-\bullet)=1-1=0$.

Formuła Riemanna-Hurwitza mówi, że jeśli mamy funkcję $f:\Conf_2(\mathbf{Y})\to \mathbf{Y}$, to wtedy
$$\chi(\Conf_2(\mathbf{Y}))=\int_{\mathbf{Y}}\chi(f^{-1}(X))d\chi(X).$$
W tym konkretnym przypadku od razu w oczy rzuca się funkcja
$$f(x, y)=x,$$
w której przeciwobrazem dowolnego punktu jest $\mathbf{Y}$ bez punktu ($\mathbf{Y}-\bullet$). Charakterystyka Eulera $\mathbf{Y}$ wynosi $1$, więc jeśli korzystamy z addytywnej definicji, to 
$$\chi(\mathbf{Y}-\bullet)=1-1=0.$$
W takim razie 
$$\chi(\Conf_2(\mathbf{Y}))=\chi(\mathbf{Y})\chi(\mathbf{Y}-\bullet)=1\cdot 0=0.$$

%Zauważmy, że teraz możemy napisać funkcję $g:\Conf_2(\mathbf{Y})\to \conf_2(\mathbf{Y})$ która "skleja" dwa punkty będące swoimi permutacjami. W takim razie nad dowolnym punktem w $\conf_2(\mathbf{Y})$ wiszą dwa punkty w $\Conf_2(\mathbf{Y})$. W takim razie 
%$$2\cdot\chi(\Conf_n(\mathbf{Y}))=\chi(\conf_n(\mathbf{Y}))$$
%i $\chi(\conf_n(\mathbf{Y}))=0$.
%\bigskip

Kolejną funkcją, którą użyjemy do liczenia $\conf_2(\mathbf{Y})$, będzie przypisanie punktowi jego orbity, czyli funkcja
$$g:\Conf_2(\mathbf{Y})\to \Conf_2(\mathbf{Y})/S_2.$$
Dowolny punkt $(x, y)\in \Conf_2(\mathbf{Y})$ posiada orbitę dwuelementową: $\{(x, y), (y, x)\}$. W takim razie, formuła Riemanna-Hurwitza mówi, że
$$\chi(\Conf_2(\mathbf{Y}))=\int_{\conf_2(\mathbf{Y})}\chi(g^{-1}(X))d\chi(X)=2\cdot\chi(\conf_2(\mathbf{Y}))$$
$$\chi(\conf_2(\mathbf{Y}))=\frac{1}{2}\chi(\Conf_2(\mathbf{Y}))=\frac{1}{2}\cdot 0=0.$$

\bigskip

$\mathbf{n=3}$ \dotfill

Rozważmy teraz funkcję 
$$f:\Conf_3(\mathbf{Y})\to \Conf_2(\mathbf{Y})$$ 
taką, że 
$$f(x, y, z)=(x, y).$$
W tym przypadku charakterystyką Eulera dowolnego punktu w $\Conf_2(\mathbf{Y})$ jest $\mathbf{Y}$ bez dwóch punktów - $z$ nie może przyjąć wartości $x$ ani $y$. Charakterystyka Eulera takiego przeciwobrazu to 
$$\chi(\mathbf{Y}-\bullet-\bullet)=\chi(\mathbf{Y}-\bullet)-1=0-1=-1.$$
Używając wzoru Riemanna-Hurwitza dostajemy
$$\chi(\Conf_3(\mathbf{Y}))=\int_{\Conf_2(\mathbf{Y})}\chi(f^{-1}(X))d\chi(X)=\chi(\Conf_2(\mathbf{Y}))\cdot\chi(\mathbf{Y}-\bullet-\bullet)=-1\cdot0=0.$$
Oczywiste jest również odwzorowanie $g:\Conf_3(\mathbf{Y})\to \conf_3(\mathbf{Y})$, w którym przeciwobrazem dowolnego punktu z $\conf_3(\mathbf{Y})$ jest dowolna permutacja współrzędnych reprezentanta wybranej orbity. Każdy reprezentant ma $3$ współrzędne, więc przeciwobraz punktu ma moc $3!=6$.

Stosując do tego odwzorowanie Riemanna-Hurwitza dostajemy
$$\chi(\conf_3(\mathbf{Y}))=\frac{1}{6}\chi(\Conf_3(\mathbf{Y}))=\frac{1}{6}\cdot 0=0.$$
\bigskip

$\mathbf{n=4}$ \dotfill

Analogicznie jak wcześniej, funkcja $f:\Conf_4(\mathbf{Y})\to \Conf_3(\mathbf{Y})$ wraz z całką Riemanna-Hurwitza mówi, że
$$\chi(\Conf_4(\mathbf{Y}))=0.$$
Ponieważ odwzorowanie $g:\Conf_4(\mathbf{Y})\to \conf_4(\mathbf{Y})$ jest $4!$-krotne, to 
$$\chi(\conf_4(\mathbf{Y}))=\frac{0}{4!}=0.$$
