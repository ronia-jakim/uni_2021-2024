\section{15.01.24 : Ciąg spektralny kompleksu podwójnego }

$(C^{p,q})_{p,q\geq0}$ to podwójny kompleks o elementach w grupie abelowej. W tym kompleksie są różniczki poziome $d_{\to}$ i różniczki pionowe $d_{\uparrow}$. Mamy kwadrat $D=d_{\to}+(-1)^pd_{\uparrow}$ i chcemy, żeby $D^2=0$, stąd możemy napisać $d_I=d_{\to}$ i $d_{II}=(-1)^pd_{\uparrow}$, wtedy $D=d_I+d_{II}$.

  Oznaczamy $T^n=Ton^n(C^{*,*})=\bigoplus_{i+j=n} C^{i,j}$, wtedy $D:T^n\to T^{n+1}$. Chcemy badać $H^n(T^*, D)$.

    Książka [ang. page] $(E_r^{p,q})_{r\geq 0, p,q\geq 0}$ o stronach $E_0$, $E_1$,..., gdzie strona $E_r$ to $(E_r^{p,q})$ z różniczką $d_R:E_r^{p,1}\to E_r^{p+r, q-r+1}$. $E_0^{p,q}=C^{p,q}$, $d_0=d_{II}$ i każdy kolejny to $E_{r+1}=H^*(E_r, d_r)$. 

    W pewnym momencie $E_r^{p,q}$ stabilizują się, bo różniczki albo idą w $0$ grupy albo z takich grup przychodzą.

    Mamy też "tylną okładkę", czyli $E_\infty^{p,q}$ trzymającą stabilne wartości we wszystkich miejscach.

    Filtracja na $T^n$ to 
    $$F^pT^n=\bigoplus_{i+j=n, i\geq p}C^{i,j}.$$
    i $F^pT^*$ jest podkompleksem $T^*$. {\large\color{red}ZDJĘCIE}
    $$T^*=F^0T^*\supseteq F^1T^*\supseteq F^2T^*\supseteq...$$
    

    Grupę $C^{p,1}$ można opisać jako $F^pT^{p,q}/F^{p+1}T^{p+q}$.

    Taka filtracja daje też filtrację homologii kompleksu totalnego $H^n(T^*)$, bo $F^pT^*\hookrightarrow T^*$, czyli istnieje $H^n(F^pT^*)\to H^n(T^*)$. Obraz tej strzałki oznaczamy $^{(p)}H^n(T^*)$. Iloraz tych homologii to $H^{(p)}H^n(T^*)/^{(p+1)}H^n(T^*)=E^{p. n-p}_\infty$

\begin{definition}[cykle]
  $Z^{p, n-p}_r$ to te $\alpha\in F^pT^n/F^{p+1}T^n$, które mają reprezentanta $a\in F^pT^n$ takiego, że $Da\in F^{p+r}T^{n+1}$.

  \begin{center}\begin{tikzpicture}
    \draw (0, -1)--(0,4);
    \draw (-1, 0)--(6, 0);
    \node at (-0.5, 3) {$q$};
    \node at (2, -0.5) {$p$};
    \draw[dashed] (0, 3)--(2, 3);
    \draw[dashed] (2, 0)--(2, 3);
    \draw[blue](2, 3)--(6, 0);
    \draw[green] (4, 0)--(4, 3.5);

    \node at (4, -0.5) {$\color{green}p+r$};
  \filldraw (4, 1.5) circle (1pt);
  \end{tikzpicture}\end{center}

  $$E_0=Z_0\supseteq Z_1\supseteq Z_2\supseteq ...\supseteq Z_N=Z_{N+1}=...=Z_\infty/$$
  $$Z_N^{p, n-p}=Z^n(F^pT^*)/Z^n(F^{p+1}T^*)$$
\end{definition}

\begin{definition}[brzegi]
  $B_r^{p, n-p}$ to te $\alpha\in F^pT^n/F^{p+1}T^n$, które mają reprezentanta $a\in F^pT^n$ takiego, że istnieje $b\in F^{p-r+1}T^{n-1}$ takiego, że $Db=a$.

  \begin{center}\begin{tikzpicture}
    \draw (0, -1)--(0, 4);
    \draw(-1, 0)--(6, 0);

    \draw (3, 0)--(3, 3);
    \node at (3, -0.5) {$p$};
    \filldraw (3, 2) circle (1pt) node [above left] {$\alpha$};
    \draw[blue] (3, 2)--(6, 0) node [midway, above right] {$a$};
    
    \draw[green] (1.5, 0)--(1.5, 3);
    \node at (1.5, -0.5) {$\color{green}p-r$};
    \draw[orange] (5.5, 0)--(1.5, 2.7) node [midway, below] {$b$};
  \end{tikzpicture}\end{center}

  $$0=B_0\subseteq B_1\subseteq B_2\subseteq ... \subseteq B_N=B_{n+!}=...=B_\infty$$
  Czyli 
  $$B_{N}^{p, n-p}=\frac{B^nT^*\cap F^pT^n}{B^nT^*\cap F^{p+1}T^n}$$
\end{definition}

\begin{definition}
  $$E_r^{p,q}=Z_r^{p,q}/B_r^{p,q}$$
  $$d_r:E_r^{p,q}\to E_r^{p+r, q-r+1}$$

  Niech $\alpha\in E_r^{p,1}$. Wybieramy reprezentanta $\alpha'\in Z_r^{p,q}$ i reprezentanta tego $\alpha'$ $a\in F^pT^n$ takiego, że $Da\in F^{p+r}T^{n+1}$ przychodzącego z definicji $Z_r^{p,q}$. Piszemy
  $$d_r\alpha=[Da]$$
\end{definition}

\begin{fact}
  Różniczka zdefiniowana wyżej jest dobrze określone i $d_r^2$.
\end{fact}

\begin{proof}
  Oczywistą częścią jest $d_r^2=0$, bo w definicji braliśmy element i nakładaliśmy na niego rózniczkę. Aby dostać $d_r^2$ robimy to nakładanie dwa razy - oczywiście dostajemy $0$.

  {\large\color{red}ZDJĘCIE dla dobrej określoności}
\end{proof}

\begin{fact}
  $$H(E_r, d_r)=E_{r+1}$$
\end{fact}

\begin{proof}
  Zaczynamy od zrozumienia, czym jest cykl w $E_r^{p,1}$ względem $d_r$. Powiedzmy, że $\alpha$ jest takie, że $d_r\alpha=0$:
  {\large\color{red}ZDJĘCIE}

  Pokazaliśmy, że jeśli $d_r\alpha=0$, to $\alpha$ jest reprezentowalnym elementem $Z_{r+1}$, czyli mamy odwzorowanie $Z_{r+1}\to E_{r+1}$. Pozostaje pokazać, że jądrem tego odwzorowania są dokładnie brzegi $B_r$.

  {\large\color{red}KOLEJNY RYSUNEK}
\end{proof}

\begin{fact}
  $$E_\infty^{p,n-p}=^{(p)}H^n(T^*)/^{(p+1)}H^n(T^*)$$
\end{fact}

\begin{proof}
  $$Z_\infty ^{p,n-p}=Z^n(F^pT^*)/Z^n(F^{p+1}T^*)$$
  $$G_\infty^{p, n-p}=\frac{B^nT^*\cap F^pT^n}{B^nT^n\cap F^{p+1}T^n}$$
  $$^{(p)}H^n(T^*)=Z^n(F^pT^*)/B^nT^*\cap F^{p}T^*$$
  $$^{(p+1)}H^n(T^*)=Z^n(F^{p+1}T^*)/B^nT^*\cap F^{p+1}T^*$$
\end{proof}

\begin{uwaga}$ $

  \begin{enumerate}
    \item Mówimy, że ciąg spektralny $E_r^{p,q}$ zbiega do homologii kompleksu totalnego $H^n(T^*)$.
    \item Zwykle najsensowaniejszą stroną jest $E_2$:
      $$E_2^{p,q}\to H^{p+1}(T^*)$$
    \item $E_0^{p,q}$ $d_0=\uparrow=d_{II}$,

      $E_q^{p,1}=H^q(E_0^{p, *}, d_{II})=H_{II}^q(E_0^p)$, $d_1=E_1^{p,q}\to E_1^{p+1, q}$
      
      $E_q^{p,q}=H_I^p(H_{II}^q(E_0^{*,*})$ i o różniczce $d_2$ nic tak przyjemnego powiedzieć nie umiemy.
  \end{enumerate}
\end{uwaga}

\subsection{Składanie funktorów}

\begin{theorem}
  Przy założeniach twierdzenia o składaniu funktorów pochodnych

  Dla każdego $A^*\in Kom^+(\mathbf{A})$ istnieje ciąg spektralny
  $$E_q^{p,q}=R^pF(R^qG(A^*))\to R^{p+1}(F\circ G)(A^*)$$
\end{theorem}

\begin{example}
\item {\large\color{red}ZDJĘCIE}

  $Sh_{\mathbf{Ab}}(E)$ -> kategoria snopów nad $E$. $\Gamma_E$ - funktor cięć globalnych nad $E$
\end{example}

\begin{proof}
  Ponoć jest interesujące nawet gdy $G=Id$. Wtedy pokażemy 
  $$(\star) \; R^pF(H^qA^*)\to R^{p+q}F(A^*).$$

  Ponieważ $A^*\cong I^*$ przez qis, to
  $$R^n(F\circ g)A^*=H^n(R(F\circ g)(A^*))=H^n(RF(RG(A^*)))=H^n(RF(G(I^*)))=R^nF(G(I^*))$$
  a z drugiej strony
  $$R^pF(R^qG(A^*))=R^pF(H^q(RG(A^*)))=R^pF(H^q(G(I^*)))$$
  jeśli $G(I^*)$ będzie nowym $A^*$, to wystarczy podstawić do $(\star)$.

  Dowód $(\star)$ polega na użyciu rezolwenty C.E. {\large\color{red}ZDJĘĘĘCIEEEEE} Tutaj użuwa się innej filtracji niż była, takiej że bierzemy wiersze.

\end{proof}
