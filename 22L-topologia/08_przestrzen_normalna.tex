\section{LEMAT URYSOHNA}

\subsection{PRZESTRZEŃ NORMALNA}
\begin{center}\large
    Przestrzeń $X$ jest przestrzenią {\color{def}NORMALNĄ} (również $T_4$), jeżeli\smallskip\\
    $(\forall\; F,G\underset{dom}\subseteq X)\;F\cap G=\emptyset$\smallskip\\
    $(\exists\;U,V\underset{otw}\subseteq X)\;U\cap V=\emptyset\;\land\;F\subseteq U\;\land\;G\subseteq V$
\end{center}
\pmazidlo
\draw[def, ultra thick] (0, 0)--(1, 1)--(2, 0)--(1, -1)--cycle;
\draw[emp, ultra thick] (3, 0)--(4, 1)--(5, 0)--(4, -1)--cycle;
\draw[emp, thick] (1, 0) circle (1.3);
\draw[def, thick] (4, 0) circle (1.3);

\node at (1, 0) {\large\color{def}F};
\node at (4, 0) {\large\color{emp}G};
\node at (0, 0.5) {\large\color{emp}U};
\node at (5, 0.5) {\large\color{def}V};
\kmazidlo

Czyli przestrzeń jest {\color{emp}normalna}, jeżeli {\color{acc}każde dwa zbiory domknięte możemy oddzielić od \\siebie rozłącznymi zbiorami otwartymi}.\medskip\\
Przestrzenie metryczne oraz przestrzenie zwarte są przestrzeniami normalnymi.

\subsection{LEMAT URYSOHNA}
%%tutaj nadzieja zaszalał z dowodem, ale nie tracił nadziei
\begin{center}\large
    Załóżmy, że przestrzeń $X$ jest normalna. Niech $F,G\underset{dom}\subseteq X$ będą rozłącznymi zbiorami domkniętymi w $X$. Wówczas:\medskip\\
    $\color{acc}(\forall\; f:X\xrightarrow{ciagla}[0,1])\; f_{\obet F}\equiv 0\;\land\; f_{\obet G}\equiv 1$\medskip\\
    {\normalsize Warunek ten jest silniejszy od normalności.}
\end{center}
\dowod
Niech $F$, $G$ będą zbiorami domkniętymi spełniającymi założenia lematu. Z normalności \\tych zbiorów możemy wziąć zbiory otwarte $U_{\frac12}$ i $V_\frac12$ takie, że $U_\frac12$ oddziela $F$ od $G$. Ponie-\\waż $U_\frac12$ jest zbiorem otwartym, to $U_\frac12^c$ jest domknięte, więc możemy oddzielić $F$ od $U_\frac12$ za \\pomocą $U_\frac14$.\smallskip\\
Ponieważ $V_\frac12$ oddziela $F$ od $G$, to $\overline U_\frac12\cap G=\emptyset$ oraz możemy utowrzyć zbiór $U_\frac34$ oddzielający $\overline U_\frac12$ od $G$ i tak dalej. \smallskip\\
Powstaje nam konstrukcja:
\pmazidlo
\draw[emp, very thick] (0, 0) circle (1);
\draw[def, very thick] (4, 0) circle (1);
\node at (0, 0) {\color{emp}$F$};
\node at (4, 0) {\color{def}$G$};
\draw[def, thick] (0, 0) circle (1.5);
\draw[emp, thick] (4, 0) circle (1.5);
\node at (1, 0.8) {\color{def}$U_\frac12$};
\node at (5, 0.8) {\color{emp}$V_\frac12$};
\draw[gr, thin] (0, 0) circle (1.3);
\node at (-0.8, 0.5) {\color{gr} $U_\frac14$};
\draw[acc, thin] (0, 0) circle (2);
\node at (1.5, 1) {\color{acc}$U_\frac34$};
\kmazidlo

Niech $\rodz D$ będzie zbiorem liczb diadycznie wymiernych (tzn postaci $\frac k{2^n}$) z przedziału $[0, 1)$. \\Wówczas zbiór
$$\{U_d\;:\;d\in \rodz D\}$$
opisuje nam powyższą konstrukcję:
$$(\forall\;d)\;F\subseteq U_d$$
$$(\forall\;d<d')\;\overline U_d\subseteq U_{d'}$$
$$(\forall\;d)\;U_d\cap G=\emptyset$$
Zdefiniujmy funkcję
$$f:X\to [0,1]$$
$$f(x)=\begin{cases}\inf\{q\in\rodz D\;:\;x\in U_q\}\quad (\exists\;q\in\rodz D)\;x\in U_q\\1\quad wpp\end{cases}$$
Zbiory otwarte na przedziale $[0,1]$ mają postać $(a, b)=[0,b)\setminus[0,a]$. Sprawdzamy ciągłość:
\begin{align*}
    f^{-1}[[0,b)] &= \{x\;:\;\inf\{q\in\rodz D\;:\;x\in U_q\}<b\}=\\
                  &=\{x\;:\;(\exists\;q<b)\;x\in U_q\}=\\
                  &=\bigcup\limits_{q<b}U_q
\end{align*}
\begin{align*}
    f^{-1}[[0,a]] &= \{x\;:\;\inf\{q\in\rodz D\;:\;x\in U_q\}\leq a\}=\\
                  &= \{x\;:\;(\exists\;q\leq a)\}=\\
                  &=\bigcap\limits_{q\leq a}U_q^c
\end{align*}
\begin{align*}
    f^{-1}[(a,b)]&=f^{-1}[[0,b)]\setminus f^{-1}[[0, a]] = \bigcup\limits_{q<b}U_q\setminus\bigcap\limits_{q\leq a}U_q^c
\end{align*}
{\large\color{def}DOCZYTAC W KLAUS JANICH "TOPOLOGIA" BO NADZIEJA POMIESZAŁ}

\subsection{TWIERDZENIE TIETZEGO}
\begin{center}\large
    Niech $X$ będzie przestrzenią normalną, a $D\subseteq X$ \\zbiorem domkniętym. Wtedy ciągłą funkcję \smallskip\\
    $f:D\xrightarrow{ciagla} \R$\smallskip\\
    możemy rozszerzyć do ciągłej funkcji\smallskip\\
    $F:X\to \R$\smallskip\\
    takiej, że $(\forall\;x\in D)\;F(x)=f(x)$
\end{center}