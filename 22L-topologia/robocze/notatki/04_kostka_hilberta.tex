\documentclass{article}

\usepackage{../../notatka}

\begin{document}\ttfamily
\section*{KOSTKA HILBERTA $[0,1]^\N$}
{\large\color{def}METRYKA NA KOSTCE HILBERTA:}\smallskip\\
\emph{chwilowo $0\notin\N$}
{\color{emp}\large$$d(x,y)=\sum\limits_{n\in\N}|x(n)-y(n)|\cdot\frac1{2^n}$$}
$C^{(a,b)}=\{x\in[0,1]^\N\;:\;x(n)\in(a,b)\}$ - wszystkie ciagi z kostki hilberta, ktore na $n$ wspolrzednej wspelniaja pewne wymagania\bigskip\\
\emph{\color{acc}Skonczone przekroje zbiorow postaci $C^{(a,b)}_n$ stanowia baze $[0,1]^\N$}.\medskip\\
\dowod
Wystarczy pokazac, ze baza topologii to suma pewnych jej elementow:
$$\forall\;x\;\forall\;U\underset{otw}\ni x\;\exists\;B\texttt{ - bazowy}\quad x\in B\subseteq U.$$
W przypadku {\color{acc}przestrzeni metrycznej} nie musimy brac kazdego zbioru otwartego (wiemy, ze kule stanowia baze)
$$\forall\;x\in[0,1]^\N\;\forall\;\varepsilon>0\;\exists\;B\texttt{ - bazowy}\quad x\in B\subseteq B_\varepsilon(x).$$
Wezmy dowolny punkt $x\in[0,1]^\N$ oraz dowolny $\varepsilon>0$. Jak ustawic te bramki, zeby $x$ przeszedl przez te bramki ale tez na pewno bycw tej kuli.\smallskip\\
W kostce Hilberta musimy odciac ogony
$$\exists\;N\in\N\quad \sum\limits_{k>N}\frac1{2^k}<\frac\varepsilon2$$
Niech dla kazdego $n\leq N$
$$I_n=(x(n)-\frac\varepsilon4, x(n)+\frac\varepsilon4)$$
Czyli na kolejnych miejscach ustawiamy bramki o srednicy $\frac\varepsilon2$. Teraz ich przekroj:
$$x\in\bigcap\limits_{n\leq N}C^{I_n}_n\subseteq B_\varepsilon(x)$$
Czemu ten przekroj jest w kuli? Wezmy element $y\in\bigcap C^{I_n}_n$ i policzby jak bardzo to jest od $x$
$$\sum\limits_{n\in\N}{|x(n)-y(n)|\over 2^n}=\sum\limits_{n\leq N}{|x(n)-y(n)\over 2^n|}+\sum\limits_{n>N}{|x(n)-y(n)|\over2^n}<\varepsilon.$$
\emph{Czyli w kostce Hilberta lepiej jest myslec o bramkach niz o kulach :c}
\kondow
\podz{gr}\bigskip\\
{\large\color{tit}WNIOSKI:}\medskip\\
\indent 1. $\{0,1\}^\N$ jest podprzestrzenia $[0,1]^\N$. Bo kule sa dokladnie takiej postaci jak przekroc $\bigcap\limits_{n\leq N}C^{I_n}_n$, czyli ustawiamy bramki na ktorys pierwszych wyrazach (bramka jest prefiks).\smallskip\\
\indent2. Topologia $[0,1]^\N$ jest topologia {\color{emp}zbieznosci punktowej}. Ciag zbiega w kostce Hilberta wtw ciagi jego wspolrzedych zbiegaja w $\R$. (to wypadaloby pokazac)\bigskip\\
\podz{tit}\bigskip
\begin{center}\large
    Niech $X$ bedzie przestrzenia metryczna i osrodkowa. Wtedy\smallskip\\
    $\exists\;Y\subseteq [0,1]^\N\quad X\cong Y$
\end{center}\bigskip
\dowod
Skoro $X$ jest przestrzenia osrodkowa, wiec
$$\{d_1, d_2,...., d_n\}\exists\;D\subseteq X$$
istnieje przeliczalny zbior gesty (kroi sie niepusto ze wszystkimi otwartymi).\smallskip\\
Zdefiniujmy funkcje
$$h:X\to[0,1]^\N$$
$$h(x)=\langle d(x,d_1), d(x,d_2), ..., d(x, d_n)\rangle$$
czyli liczymy pokolei odleglosci od kolejnych elementow naszego zbioru gestego.\smallskip\\
Jesli mamy przestrzen metryczna, to mozemy zalozyc, ze punkty tej przestrzeni sa odlegle od siebie o mniej niz jeden (tw sprzed dwoch tyg.). Zakladamy wiec, bez zmniejszenia ogolnosci,
$$\forall\;x,y\in X\quad d(x,y)\leq 1$$
Dlaczego ta funkcja jest 1-1?
\pmazidlo
\node (x) at (0,0) {x};
\node (y) at (3, 2) {y};
\draw [white, thick] (x)--(y);
\node (dn) at (0, 0.5) {$d_n$};
\draw [acc, thick] (0, 0) circle (1);
\kmazidlo
Musimy znalezc taki element zbioru gestego, ze odleglosc od $x$ i $y$ jest rozna. Czyli bierzemy sobie odleglosc miedzy $x$ i $y$ i rysujemy w $x$ kule o promieniu 100 razy mniejszym - wtedy $d(x, d_n)$ musi byc mniejsza od promienia tej kuli, a $d(y, d_n)$ musi byc wieksza.\medskip\\
Dlaczego ta funkcja jest na?\smallskip\\
Nie chcemy tego pokazywac - to by bylo, ze kazda przestrzen metryczna jest homeomorf z kostka Hilberta. Nam jest potrzebny tylko $h[X]=Y$.\smallskip\\
Wystarczy pokazac, ze $h$ jest ciagla i ze $h^{-1}$ tez jest ciagla. Wezmy zbiory bazowe
$$h^{-1}[C^{(a,b)}_n]$$
i pokaze ze to jest otwarte. Wtedy przeciwobrazy skonczonych przekroi takich zbiorow tez jest otwarty.
$$C^{(a,b)}_n=\{x\in X\;:\;d(x,d_n)\in (a,b)\}$$
\pmazidlo
\node (d) at (0.3, 0.3) {$d_n$};
\draw[white, thick] (0,0)--(-0.65, 0.75);
\draw[white, thick] (0,0)--(0, 1.5);
\node at (-0.4, 0.1) {a};
\node at (0.2, 1.3) {b};
\node (c) at (0,0) {$\bullet$};
\draw[acc, thick] (0,0) circle(1);
\draw[emp, thick] (0,0) circle(1.5);
\kmazidlo
I to jest zbior otwarty.\\
Teraz zakladamy, ze $h^{-1}$ jest ciagla i chcemy pokazac, ze przeciwobrazy zbiorow otwartych sa otwarte:
$$h[B_\varepsilon(x)]\ni y=h(x).$$
Zeby uniknac technicznych rzeczy, bierzemi $y=h(x)$, ale on wcale taki byc nie musi.
\pmazidlo
\node (x) at (0.1,0.2) {x};
\node at (0,0) {$\bullet$};
\node (dn) at (-0.3, -0.15) {$d_n$};
\draw[white, thick] (0,0) circle (0.5);
\draw[emp, thick] (0, 0) circle (1.5);
\node (y) at (3.1, 2.2) {y};
\node at (3,2) {$\bullet$};
\draw[acc, ultra thick] (y) circle (1);
\draw[def, ultra thick, ->] (0,0)--(3, 2);
\draw[white, thick] (0,0)--(0.2,-1.45);
\node at (0.4, -0.7) {$\varepsilon$};
\kmazidlo
Patrze na kule $B_\varepsilon(x)$ i o wiele mniejsza kule $B_{\frac\varepsilon{10}}(x)$. W niej znajuje $d_n$. Za pomoca tego $d_n$ zdefniuje bramke, czyli chce wszystko to, co jest blizej $y$ niz $\frac\varepsilon{10}$:
$$y\in C_n^{(y(n)-\frac\varepsilon{10}, y(n)+\frac\varepsilon{10})}\subseteq h[b_\varepsilon(x)]$$
Chce pokzazac, ze 
$$z=h(x)\in C_n^{(y(n)-\frac\varepsilon{10}, y(n)+\frac\varepsilon{10})}$$
No ale teraz wiem, ze
$$|z(n)-y(n)|<\frac\varepsilon{10}$$
Odleglosc $v$ i $x$ od $d_n$ i ich roznica nie jest wieksza niz $\frac\varepsilon{10}$. W takim razie
$$d(x,v)<\frac\varepsilon5$$
Ale wtedy tym bardziej $v\in B_\varepsilon(x)$, co oznacza, ze
$$h(v)\in h[B_\varepsilon(x)]$$
{\color{emp}Nadzieja wszystko pomieszal}
\kondow
\subsection*{ZWARTOSC}
\begin{center}\large
    {\color{def}POKRYCIE }- rodzina zbiorow otwartych pokrywajaca $X$\smallskip\\
    $\bigcup \mathcal{U}=X$
\end{center}
\begin{center}\large
    Przestrzen topologiczna $X$ jest {\color{def}ZWARTA}, gdy z kazdego pokrycia mozna wybrac podpokrycie skonczone.
\end{center}\bigskip
$(0,1)$ w metryce euklidesowej nie jest zwarty. Nie mozemy wybierac takich przedzialikow otwartych, ze one ladnie zbiegaja do 1 (jesli wyrzucimy jeden zbior to to juz nie jest pokrycie).\medskip\\
$[0,1]$ w metryce euklidesowej jesli dzielimy znowu na coraz to mniejsze przedzialy otwarte, to zawsze zostaje ten malutki i jak go sobie wybiore, to cala reszta moze byc wyrzucona i jest pokrycie.\bigskip
\begin{center}\large
    Przestrzen metryczna jest {\color{def}ZWARTA }wtedy i tylko wtedy, gdy z kazdego ciagu mozemy wybrac podciag zbiezny.
\end{center}\bigskip
\dowod
$\implies$\medskip\\
Wybiezmy dowolny ciag $(x_n)$. Mamy dwie sytuacje:\smallskip\\
\indent 1. $\exists\;x\in X\;\forall\;\varepsilon>0 B_\varepsilon(x)\cap A$ jest nieskonczony, czyli nieskonczenie wiele wyrazow wpada do naszej kuli (jest to \color{emp} PUNKT SKUPIENIA CIAGU\color{txt}).\smallskip\\
Wtedy wybieramu $(n_k)$ rosnacy taki, ze
$$B_{n_k}\in B_\frac1k(x)$$
Taki ciag nie ma wyboru:
$$(x_{n_k})\to x$$
\indent 2. Zalozmy, ze $(x_n)$ nie ma punktu skupienia.\smallskip\\
Wezmy dowolny $x\in X$. Istnieje $B_x$, czyli kula o srodku $x$, ktora zawiera tylko skonczenie wiele wyrazow naszego ciagu.
$$\{B_x\;:\;x\in X\}$$
jest pokryciem. To w takim razie istnieje $F$ skonczony taki, ze
$$\{B_x\;:\;x\in F\}$$
jest pokrycim $X$. Ale to jest sprzecznosc, bo wtedy mamy skonczony ciag.\bigskip\\
$\impliedby$\medskip\\
Nadziei sie nie chce i to pominie
\kondow
Kazdy ciag ograniczony na prostej ma podciag zbiezy - czyje to? {\color{emp}(Boltzana-Weiestrassa)} - dowodzi czemu $[0,1]$ jest zwarty (czyli ma pokrycie).

\end{document}