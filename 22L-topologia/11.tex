\section{PRZESTRZENIE ZUPEŁNE}
\subsection{CIĄG CAUCHY'EGO}
\begin{center}\large
    Ciąg $(x_n)$ jest Chauchy'ego, jeśli\smallskip\\
    $\color{def}(\forall\;\varepsilon>0)(\exists\;N)(\forall\;n,m\geq N)\; d(x_n, x_m)<\varepsilon$\medskip\\
    Jeśli ciąg jest zbieżny, to \\jest Cauchy'ego, ale nie musi być odwrotnie.
\end{center}
\begin{center}\large
    Przestrzeń metryczna jest {\color{def}ZUPEŁNA}, \\gdy każdy ciąg Cauchy'ego jest zbieżny
\end{center}
Przykłady metryk gdzie ciągi Cauchyego mogą być niezbieżne:\smallskip\\
\indent1. $(0,1)$ z ciągiem $(\frac1n)$\medskip\\
\indent2. $\Q$ wiele ciągów Cauchyego, które zbiegają do $a\in\R$, czyli nie są zbieżne w $\Q$.\medskip\\
{\large\color{acc}PRZYKŁADY}\smallskip\\
\indent1. przestrzeń dyskretna\smallskip\\
\indent2. $[0,1]$\smallskip\\
\indent3. $\R$, bo jeśli mamy ciąg Cauchyego, to on musi być ograniczony, czyli 
$$(\exists\;a,b)(\forall\;n)\;x_n\in[a,b]$$ 
i stosujemy zwartość $[a,b]$ (fakt niżej)\bigskip\\

\begin{center}\large
    \color{acc}Przestrzenie zwarte są zupełne.
\end{center}
\dowod
Jeśli $(x_n)$ ma podciąg zbieżny i jest Cauchyego, to jest zbieżny. (więcej na liście zadań)\bigskip\\
\podz{gr}\bigskip\\
Zupełność nie jest własnością topologiczną - $(0,1)$ jest homeomorficzne z $\R$, ale $(0,1)$ nie jest zupełna, a $\R$ jest zupełn - {\large\color{acc}NIE ZACHOWUJE SIĘ PRZEZ HOMEOMORFIZMY}\bigskip
\begin{center}\large
    Przestrzeń metryczna jest {\color{def}METRYZOWALNA W SPOSÓB ZUPEŁNY}, gdy\smallskip\\
    $(\exists\;Y)\;X\cong Y\quad Y\;-\;zupelna$
\end{center}

\subsection{TWIERDZENIE BANACHA}
\begin{center}\large
    {\color{def}TWIERDZENIE BANACHA O PUNKCIE STAŁYM}\smallskip\\
    jeśli $(X,d)$ jest przestrzenią zupełną i mamy \smallskip\\
    $f:X\to X$\smallskip\\
    która jest {\color{acc}kontrakcją} (tzn $(\exists\;c<1)(\forall\;x,y\in X)\;d(f(x), f(y))\leq d(x,y)\cdot c$)\smallskip\\
    to $\color{def}(\exists\;x\in X)\;f(x)=x$
\end{center}
\dowod
Tworzymy ciąg w następujący sposób (iterujemy $f$):
$$x_0\in X$$
$$x_{n+1}=f(x_n).$$
Będziemy się starali pokazać, że jest to ciąg Cauchyego. Pomiędzy $x_0, x_1$ odglełość jest średnio kontrolowana, ale już odległość
$$d(x_1, x_2)=d(f(x_0), f(x_1))\leq c\cdot d(x_0, x_1).$$
Robimy tak $n$ razy i dostajemy
$$d(x_n, x_{n+1})=d(f(x_{n-1}), f(x_n))\leq c^n\cdot d(x_0, x_1),$$
czyli kolejne wyrazy ciągu są coraz bliżej siebie.
\begin{align*}
d(x_n, x_m)&\leq d(x_n, x_{n+1})+...+d(x_{m-1}, x_m)\leq \\
            &\leq c^n d(x_0, x_1)+...+c^md(x_0, x_1)=\\
            &=d(x_0, x_1)(x^n+...+c^m)=\\
            &=d(x_0, x_1)\sum c^k
\end{align*}
Ale wtedy
$$(\forall\;\varepsilon>0)(\exists\;N)\sum\limits_{n=N}^\infty c^n d(x_0, x_1)<\varepsilon,$$
bo $\sum c^n<\infty$. Wtedy dla $n,m>N$
\begin{align*}
    d(x_n, x_m)&\leq (c^n+...+c^m)d(x_0, x_1)\leq \\
                &\leq\sum\limits_{k=N}^\infty c^kd(x_0, x_1)<\varepsilon
\end{align*}
Z zupełności $(\exists\; x)\;(x_n)\to x$, czyli $x=\lim x_n$. W takim razie
$$f(x)=\lim f(x_n)=\lim x_{n+1}=x$$
\kondow

\subsection{TWIERDZENIE CANTORA}
\begin{center}\large
    {\color{def}TWIERDZENIE CANTORA}: jeśli $(X, d)$ jest przestrzenią zupełną, a $(F_n)$ to ciąg zbiorów domkniętych taki, że\smallskip\\
    $diam(F_n)\to 0$, gdzie $diam(F) = \sup\{d(x, y)\;:\;x,y\in F\}$,\smallskip\\
    oraz $(\forall\;n)\;F_{n+1}\subseteq F_n$.\medskip\\
    {\color{acc}Wtedy $\bigcap F_n\neq\emptyset$.}
\end{center}
\dowod
Chcemy wskazać przynajmniej jeden element tego, co chcemy pokazać że jest niepuste. \\Skonstruujmy więc ciąg
$$x_0\in F_0$$
$$x_1\in F_1$$
$$x_2\in F_2$$
$$x_n\in F_n.$$
Sprawdźmy, że jest to ciąg Cauchyego. Niech $\varepsilon>0$, a $N$ będzie takie, że
$$(\forall\;n>N)\;diam(F_n)<\varepsilon.$$
To wtedy
$$(\forall\;n,m>N)\;x_n, x_m\in f_N.$$
Czyli istnieje granicaL $x=\lim (x_n)$. Jeśli należy do przekroju, to należy do wszystkich zbiorów:
$$x\in F_0,\;bo\;(\forall\;n)\;x_n\in F_0$$
$$x_m\in F_n,\;bo\;(\forall\;m>n)\;x_m\in F_n$$
czyli $x\in\bigcap\limits_{n\in\N} F_n$.
\kondow

\subsection{TWIERDZENIE BARE'A}
\begin{center}\large
    {\color{def}TWIERDZENIE BARE'A}: jeśli $(X,d)$ jest przestrzenią \\{\color{acc}metryzowalną w sposób zupełny}, to jeśli $(F_n)$ \\jest ciągiem zbiorów {\color{acc}domkniętych o pustym wnętrzu}, to\smallskip\\
    $\bigcup F_n\neq X$
\end{center}
\dowod
Załóżmy, że $X$ jest zupełna. Będziemy konstruować ciąg Cauchyego, który zbiega do punktu spoza $\bigcup F_n$.
$$x_0,\;r_0<1\quad \overline{B(x_0, r_0)}\cap F_0=\emptyset,$$
czyli wybieram punkt z dopełnienia $F_0$ i opisuję na nim kulę, która tnie się pusto z $F_0$.
$$x_1\in B(x_0, r_0), \;r_1<\frac12\quad \overline{B(x_1, r_1)}\cap F_1=\emptyset,$$
co jest możliwe, bo $F_1$ ma puste wnętrze.\bigskip\\
$(x_n)$ jest Cauchyego, bo $r_n\to0$. Z zupłności $(\exists\;x)\;x=\lim(x_n)$. Chcę pokazać, że $x\notin \bigcup F_n$.
$$x\notin F_0,\;bo\;(x_n)\subseteq \overline{B(x_0,r_0)}$$
czyli $x\notin\bigcup F_n$.
\kondow
{\large\color{acc}WNIOSKI}\medskip\\
\indent 1. $|\R|\neq|\N|$\smallskip\\
\indent 2. $\Q$ nie jest metryzowalne w sposób zupełny, bo nie spełnia twierdzenia Bare'a, \\bo $\Q=\bigcup_{q\in\Q}\{q\}$.\bigskip\\
\podz{gr}\bigskip
\begin{center}\large
    {\color{acc}Istnieje funkcja ciągła $f:[0,1]\to\R$ \\niemonotoniczna na żadnym przedziale}
\end{center}
\dowod
Niech $I$ będzie przedziałem, a $C^I_\nearrow$ niech będzie zbiorem funkcji ciągłych na $[0,1]$, które \\są niemalejące. Pokażemy, że jest to zbiór domknięty o pustym wnętrzu. {\color{cyan}????}

\subsection{NIE WIEM CO SIĘ DZIAŁO}
\begin{center}\large
    $C[0, 1]$ z metryką supremum jest przestrzenią zupełną.
\end{center}
\dowod
Niech $(f_n)$ będzie ciągiem Cauchyego funkcji ze zbioru $C[0, 1]$. Wówczas
$$(\forall\;\varepsilon>0)(\exists\;N)(\forall\;n, m>N)d_sup(f_n, f_m)<\varepsilon.$$
Ustalmy dowolne $x\in[0,1]$ takie, że $(f_n(x))_n\subseteq \R$ jest ciągiem Cauchyego. Ustalmy dowolne $\varepsilon>0$. Z tego, że jest to ciąg Cauchyego wiemy, że istnieje $N$ takie, że
$$(\forall\;n, m>N)\;|f_n(x)-f_m(x)|<\varepsilon.$$
Zatem $(f_n(x))$ ma granicę. Niech $f$ będzie taką funckją, że\smallskip\\
\indent - $f$ jest granicą $(f_n)$, czyli
$$(\forall\;\varepsilon>0)(\exists\;N)(\forall\;n>N)(\forall\;x\in[0, 1])\;|f_n(x)-f(x)|<\varepsilon$$
\indent - $f$ jest ciągłe - bo jest jednostają granicą ciągu funkcji ciągłych.
Czyli pokaza-\\liśmy, że dowolny ciąg $(f_n)$ jest jednostajnie zbieżny do pewnego $f$, więc przestrzeń jest zupełna {\color{cyan}????}
\kondow
Przestrzeń $C[0, 1]$ z metryką całkową nie jest zupełna, bo w ciągu Cauchyego pola dążą do 0, wystarczy ograniczyć je prostokątami - nie ma ciągłości. Jeśli z kolei zdefiniujemy metrykę korzystając z {\color{acc}całki Lebesgue'ga}, co znajduje zastosowanie w rachunku prawdo-\\podobieństwa.\bigskip\\
\podz{gr}\bigskip
\begin{center}\large
    Jeśli $X$ jest przestrzenią zupełną, a $F_n\subseteq X$ jest zbiorem domkniętym o pustym wnętrzu, czyli $\bigcup F_n\neq X$. Wtedy\smallskip\\
    {\color{def}$A\subseteq X$ jest 1 kategorii} (Bar{\"e}'a), gdy\smallskip\\
    $A=\bigcup F_n$\smallskip\\
    dla pewnych $F_n$ domkniętych o pustym wnętrzu.
\end{center}

\subsection{TWIERDZENIE ASOLIEGO-ARZELI}
\begin{center}\large
    Niech $F\subseteq C[0, 1]$ takie, że\smallskip\\
    \begin{tabular} { c }
        \makecell [l] {
            - $F$ jest wspólnie ograniczony, czyli\smallskip\\
            \indent{\normalsize$(\exists\;c>0) (\forall\;f\in F)(\forall\;x\in [0, 1])|f(x)|<c$}\smallskip\\
            - $F$ są jednakowo ciągłe, czyli\smallskip\\
            \indent{\normalsize$(\forall\;x)(\forall\;\varepsilon>0)(\exists\;\delta>0)(\forall\;f\in F)\;|x-y|<\delta\implies |f(x)-f(y)|<\varepsilon$} }
    \end{tabular}\medskip\\
    Wtedy $\overline F$ jest podprzestrzenią zwartą (\emph{jest domknięte}).
\end{center}
\dowod
Korzystając z faktu udowodnionego na ćwiczeniach: zupełnośc i całkowita ograniczoność $\implies$ zwartość, a ta z kolei pociąga zupełność.\bigskip\\
Wiemy, że przestrzeń $C[0, 1]$ jest przestrzenią zupełną oraz że $\overline F$ jest domknięciem prze-\\strzeni zupełnej, więc też jest zupełne.\smallskip\\
Chcemy miec całkowitą ograniczoność, czyli
$$(\forall\;\varepsilon>0)(\exists\;A\;-\;\texttt{skończone})(\exists\;a\in A)\;d(x, a)<\varepsilon.$$
Ustalmy dowolne $\varepsilon>0$. Wówczas rodzina
$$\rodz A=\{U\underset{otw}\subseteq C[0, 1]\;:\;(\forall\;f\in F)\;diam( f[U])<\varepsilon\}$$
jest pokryciem, ponieważ dla dowolnego $x\in [0, 1]$ istnieje $\delta$ dobrana do $\frac\varepsilon2$ z jednakowej \\ciągłości:
$$(\forall\;f\in F)\;diam(f[B_j(x)])<\varepsilon$$
$$x\in B_\delta(x)\in \rodz A.$$
{\color{cyan}NIE ROZUMIEM TEGO DOWODU}

\subsection{PRZESTRZEŃ POLSKA}
\begin{center}\large
    $X$ jest {\color{def}PRZESTRZENIĄ POLSKĄ}, jeśli $X$ jest {\color{acc}zupełna} i {\color{acc}ośrodkowa}.
\end{center}\bigskip
{\color{cyan}NIE ROZUMIEM TUTAJ RESZTY}\bigskip

\subsection{ZBIORY I KATEGORII}
\begin{center}\large
    Zbiór $M\subseteq X$ jest zbiorem {\color{def}I KATEGORII}, jeżeli $M\subseteq \bigcup\limits_{n\in\N}F_n$, \\gdzie $F_n$ to domknięte zbiory o pustym wnętrzu.
\end{center}\bigskip
\begin{center}\large
    $A\subseteq X$ ma {\color{def}WŁASNOŚĆ BAIRE'a}, jeśli istnieje zbiór otwarty $U$ oraz $M$ I kategorii takie, że\smallskip\\
    $A=U\Delta M=(U\setminus M)\cup(M\setminus U)$\medskip\\
    czyli jest otwarty modulo zbiór I kategorii
\end{center}\bigskip
\podz{gr}\bigskip
\begin{center}\large
    Jeżeli $F\subseteq \R$ jest domknięty, to ma własność Baire'a
\end{center}
\dowod
$$F=Int(F)\cup(F\setminus Int(F))$$
Zauważmy, że zbiór $(F\setminus Int(F))$ jest domkniętym zbiorem o pustym wnętrzu, a $Int(F)$ jest \\zbiorem otwartym. W takim razie $F$ spełnia własnośc Baire'a ($F=(Int(F)\setminus Bd(F))\cup(Bd(F)\setminus Int(F))$)
\kondow
Zbiory I kategorii również mają własność Baire'a.\bigskip\\
\begin{center}\large
    Rodzina zbiorów o własności Baire'a jest \\zamknięta na dopełnienia i nieskończone sumy
\end{center}
\dowod
Niech zbiór $A$ będzie zbiorem spełniającym własność Baire'a, czyli
$$A=U\Delta M.$$
W takim razie
$$A^c=Int(A^c)\Delta (M\cup((U^c)\setminus Int(U^c))),$$
gdzie $(U^c)\setminus Int(U^c)$ jest zbiorem domkniętym o pustym wnętrzu. 