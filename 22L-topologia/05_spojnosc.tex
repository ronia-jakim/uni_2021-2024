\section{SPÓJNOŚĆ}
\begin{center}\large
    Przestrzeń $X$ {\color{def}NIE JEST SPÓJNA} jeśli istnieją \\otwarte niepuste $U,v\subseteq X$ takie, że\smallskip\\
    $U\cap V=\emptyset\;\land\; U\cup V=X$
\end{center}
Przestrzeń $[0, 1]$ jest spójna.\bigskip\\
\dowod
Załóżmy nie wprost, że istnieją dwa rozłączne zbiory otwarte $U, V\subseteq [0,1]$ takie, że \\$U\cup V=[0,1]$. Załóżmy, bez straty ogólności, że $0\in U$.\smallskip\\
Niech $b\in \inf V$. Rozważmy dwa przypadki:\medskip\\
\indent 1. $b\in V$\smallskip\\
Wówczas każde otoczenie $b$ kroi się niepusto z $U$, czyli nie należy do $V$, a więc $V$ nie \\jest zbiorem otwartym.\medskip\\
\indent 2. $b\notin V$\smallskip\\
Wtedy $U$ nie jest otwarty, bo $b\in U$ i otoczenia $b$ kroją się niepusto z $V$, więc one nie należa od $U$, a więc $U$ nie jest otwarte.\medskip\\
Mamy sprzecznośc w obu przypadkach, więc nie istnieją rozłączne zbiory $U, V\underset{otw}\subseteq [0, 1]$ takie, że $U\cup V=[0,1]$. W takim razie $[0,1]$ jest spójne.
\kondow
\podz{gr}\bigskip
\begin{center}\large
    Jeśli $(X_i)_{i\in I}$, $X_i\subseteq X$, to przestrzenie spójne oraz\smallskip\\
    $\bigcap X_i\neq \emptyset$,\smallskip\\
    to wówczas\smallskip\\
    $\bigcup X_i$ jest przestrzenią spójną.
\end{center}\bigskip
\dowod
{\color{cyan}nie wiem co się zadziało}

\subsection{ŁUKOWA SPÓJNOŚĆ}
\begin{center}\large
    Przestrzeń $X$ jest {\color{def}ŁUKOWO SPÓJNA} jeśli\smallskip\\
    $(\forall\;x\neq y\in X)(\exists\;h:[0, 1]\xrightarrow[ciagla]{} X)\; h(0)=x\;\land\;h(1)=y$
\end{center}\bigskip
Przeliczalne zbiory nie są łukowo spójne. Przeliczalne zbiory Hausdorffa nie są spójne.\bigskip
\begin{center}\large
    Łukowa spójność pociąga za sobą spójność.
\end{center}
\dowod
Załóżmy, nie wprost, że przestrzeń $X$ jest łukowo spójna i jest niespójna, czyli is-\\tnieją dwa rozłączne zbiory otwarte $U, V\subseteq X$ takie, że $U\cup V=X$. Niech $x\in U$ oraz $y\in V$. Ze spójności łukowej wiemy, że
$$(\exists\;h:[0,1]\to X)\;h(0)=x\;\land\;h(1)=y.$$
Spróbujemy znaleźć punkt w obrazie $h$, który nie należy ani do $U$ ani do $V$.\medskip\\
Niech 
$$h[[0, 1]]=I\cong [0, 1]$$
wtedy 
$$x\in I\cap U\neq\emptyset$$
$$y\in I\cap V\neq\emptyset.$$
Zatem $I$ nie jest spójne, bo możemy je wyrazić jako $(I\cap U)\cup(I\cap V)$, co jest sprzeczne{\color{cyan}CZEMU?}
\kondow
\subsection{KOŁO WARSZAWSKIE i przyjaciele}
Czy twierdzenie wyżej działa w drugą stronę? Nie, spójrzmy na przestrzeń nazwaną {\color{acc}Ko-\\łem warszawskim}:
$$X=\{\sin\frac1x\;:\;x\in\R\}\cup\{0\}\times[-1, 1]$$
\pmazidlo
    \begin{axis}[
        axis x line=middle,
        axis y line=middle,
        enlarge y limits=true
    ]
        \addplot[
            domain=0.001:1, 
            gr, 
            ultra thick,
            samples=200
        ] {sin(deg(1/(x)))};
        \addplot[
            domain=-1:0.001, 
            gr, 
            ultra thick,
            samples=200
        ] {sin(deg(1/(x)))};
        \addplot +[
            mark=none,
            acc,
            ultra thick
        ] coordinates {(0, -1) (0, 1)};
        \addplot +[
            mark=none,
            acc,
            ultra thick
        ] coordinates {(0.01, -1) (0.01, 1)};
        \addplot +[
            mark=none,
            acc,
            ultra thick
        ] coordinates {(-0.01, -1) (-0.01, 1)};
    \end{axis}
\kmazidlo
Jest to przestrzeń spójna, ale nie jest spójna łukowo. Jeśli zostanie tylko $\{\sin\frac1x\;:\;x\in\R\}$, to przestrzeń staje się spójna łukowo.