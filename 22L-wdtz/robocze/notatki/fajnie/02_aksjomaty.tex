\section{AKSJOMATY}
{\color{emp}Zbiór oraz należenie} uznajemy za {\color{emp}pojęcia pierwotne}, więc nie definiujemy ich tylko opi-\\sujemy ich własności.

\subsection{AKSJOMAT EKSTENSJONALNOŚĆI}
\begin{center}
    zbiór jest jednoznacznie wyznaczony przez swoje elementy\smallskip\\
    $(\forall\;x)\;(\forall\;y)\;(x=y\iff(\forall\;z)\;(z\in x\iff z\in y))$
\end{center}\medskip
Od tego momentu zakładamy, że \emph{\color{emp}istnieją wyłącznie zbiory}. Nie ma nie-zbiorów. Naszym \\celem jest budowanie uniwersum zbiorów i okazuje się, że w tym świecie można zinter-\\pretować całą matematykę.

\subsection{AKSJOMAT ZBIORU PUSTEGO}
\begin{center}
    istnieje zbiór pusty \O\smallskip\\
    $(\exists\;x)(\forall\;y)\neg\;y\in x$
\end{center}
Na podstawie {\color{def}aksjomatu ekstensjonalności} oraz {\color{def}aksjomaty zbioru pustego} można udowodnić, że istnieje {\color{emp}dokładnie jeden zbiór pusty}.\medskip\\
\begin{tabular} {m{3cm} m{15 cm}}
1. istnienie: & aksjomat zbioru pustego\\
 \\
\makecell[tl]{2. jedyność:} & \makecell[tl]{niech $P_1, P_2$ będą zbiorami pustymi. Wtedy dla dowolnego $z$ zachodzi \\$\neg\;z\in P_1\land \neg\;z\in P_2$, czyli $z\in P_1\iff z\in P_2$. Wobec tego, na mocy aksjomatu \\ekstensjonalności mamy $P_1=P_2$.}
\end{tabular}\bigskip\\
Przyjrzyjmy się następującemy systemowi algebraicznemu:
$$\rodz A_1=\parl\N\cap[10, +\infty),<\parr$$
W systemie spełnione są oba te aksjomaty:
$$\rodz A_1\models A_1+A_2$$
Ponieważ {\color{acc}nie mamy podanej interpretacji}, a nasze aksjomaty są spełnione, to spełnione \\są dla {\color{acc}dowolnej interpretacji}.

\subsection{AKSJOMAT PARY}
\begin{center}\large
    dla dowolnych zbiorów $x, y$ istnieje para $\{x, y\}$\smallskip\\
    $(\forall\;x,y)\;(\exists\;z)\;(\forall\;t)\;(t\in z\iff t=x\lor t=y)$
\end{center}
{\color{acc}Para nieuporządkowana jest jednoznacznie wyznaczona}. Aksjomat mówi tylko o istnieniu \\$z$, a można łatwo udowodnić, korzystając z aksjomatu ekstencjonalności, że takie $z$ is-\\tnieje tylko jedno.\medskip\\
Niech $P_1, P_2$ będa parami nieuporządkowanymi $x, y$. W takim razie jesli $t\in P_1$, to $t=x\lor t=y$. Tak samo $t\in P_2\iff t=x\lor t=y$. Czyli $P_1=P_2$ bo posiadają te same elementy. \bigskip\\
\podz{emp}\bigskip\\
{\color{def}SINGLETONEM} elementu $x$ nazywamy zbiór $\{x\}:=\{x, x\}$\bigskip
\begin{center}\large
    {\color{def}PARĄ UPORZĄDKOWANĄ} (wg. Kuratowskiego) \\elementów $x$ i $y$ nazyway zbiór:\smallskip\\
    $\parl x,y\parr := \{\{x\}, \{x,y\}\}$
\end{center}\medskip
\podz{gr}\medskip\\
Dla dowolnych elementów $a, b, c, d$ zachodzi:
$$\parl a, b\parr = \parl c,d\parr \iff a=c\land b=d$$
\dowod
Rozważmy dwa przypadki:\medskip\\
\indent 1. $a=b$\\
$$\parl a,a\parr = \{\{a\}, \{a, a\}\} = \{\{a\}\}$$
Czyli jeśli $x\in \{\{a\}\}$, to $x=\{a\}$. Z drugiej strony mamy 
$$\parl c, d\parr=\{\{c\}, \{c,d\}\}$$
A więc jeśli $x\in \{\{c\}, \{c,d\}\}$, to $x=\{c\}$ lub $x=\{c, d\}$. W takim razie mamy $\{a\}=\{c\}=\{c, d\}$, a więc z aksjomatu ekstensjonalności, $a=c=d$.\medskip\\
\indent 2. $a\neq b$
$$\parl a, b\parr = \{\{a\}, \{a, b\}\}$$
Jeśli więc $x\in \parl a, b\parr$, to $x=\{a\}$ lub $x=\{a, b\}$. Z drugiej strony mamy
$$\parl c, d\parr=\{\{c\}, \{c, d\}\}$$
Jeśli $x\in \parl c,d\parr$, to $x=\{c\}$ lub $x=\{c, d\}$. W takim razie otrzymujemy $\{c\}=\{a\}$ i $\{c, d\}=\{a, b\}$. Z aksjomatu ekstensjonalności mamy $a=c$ oraz $d=b$.
\kondow
\subsection{AKSJOMAT SUMY}
\begin{center}\large
    Dla dowolnego zbioru istnieje jego suma\smallskip\\
    $(\forall\;x)\;(\exists\;y)\;(\forall\;z)\;(z\in y\iff (\exists\;t)\;(t\in x\land z\in t))$
\end{center}\bigskip
Ponieważ wszystko w naszym świecie jest zbiorem, to \emph{\color{emp}każdy zbiór możemy postrzegać ja-\\ko rodzinę zbiorów} - jego elementy też są zbiorami. W takim razie suma tego zbioru to \\suma rodziny tego zbioru.\medskip\\
{\color{def}Suma jest określona jednoznacznie} i oznaczamy ją $\bigcup x$.\bigskip\\
\dowod
Załóżmy nie wprost, ze istnieją dwie sumy zbioru $x$: $S_1$ i $S_2$. Wtedy
$$(\forall\;z)(z\in S_1\iff (\exists\;t\in x) (z\in t))$$
$$(\forall\;z)(z\in S_2\iff (\exists\;t\in x) (z\in t))$$
Zauważamy, że
$$z\in S_1\iff (\exists\;t\in x)z\in t\iff z\in S_2$$
a więc $S_1$ i $S_2$ mają dokładnie te same elementy, więc z aksjomatu ekstencjonalności są \\tym samym zbiorem.
\kondow
Suma dwóch zbiorów:
$$x\cup y := \bigcup\{x, y\}$$
\dowod
Ustalmy dowolne $z$. Wtedy mamy
\begin{align*}
    z\in \bigcup\{z, y\}&\overset{4}\iff (\exists\;t)\;(t\in \{x, y\}\land z\in t)\overset{3}\iff (\exists\;t)((t=x\lor t=y)\land z\in t)\iff\\
    &\iff (\exists\;t)\;((t=x\land z\in t)\lor (t=y\land z\in t))\iff \\
    &\iff (exists\;t)(t=x\land z\in t)\lor(\exists\;t)(t=y\land z\in t)\implies\\
    &\implies (\exists\;t)(z\in x)\lor (\exists\;t)(z\in y\iff z\in x\lor z\in y)
\end{align*}
\kondow
\subsection{AKSJOMAT ZBIORU POTĘGOWEGO}
\begin{center}\large
    dla każdego zbioru istnieje jego zbiór potęgowy\smallskip\\
    $(\forall\;x)(\exists\;y)(\forall\;z)z\in y\iff (\forall\;t\in z) t\in x$\smallskip\\
    $(\forall\;x)(\exists\;y)(\forall\;z) \;z\in y\iff z\subseteq x$
\end{center}\bigskip
Zbiór potęgowy jest wyznaczony jednoznacznie i oznaczamy go $\Po x$\medskip\\
\dowod
Załóżmy, nie wprost, że istnieją dwa różne zbiory potęgowe $P_1$ i $P_2$ dla pewnego zbioru \\$x$. Wówczas
$$(\forall\;z)\;z\in P_1\iff z\subseteq x$$
$$(\forall\;z)\;z\in P_2\iff z\subseteq x$$
Zauważamy, że
$$z\in P_1\iff z\subseteq x\iff z\in P_2,$$
czyli zbiory $P_1$ i $P_2$ mają dokładnie te same elementy, więc na mocy aksjomatu ekstencjo-\\nalności $P_1=P_2$
\kondow
\subsection{AKSJOMAT WYRÓŻNIANIA}
To tak naprawdę schemat aksjomatu, czyli nieskończona rodzina aksjomatów
\begin{center}\large
    {\color{def}SIMPLIFIED VERSION:} niech $\varphi(t)$ będzie formułą języka teorii mnogości. Wtedy dla tej formuły mamy $\color{tit}A_{6\varphi}$ dla każdego zbioru $x$ istnieje zbiór, którego elementy spełniają własność $\varphi$\smallskip\\
    $(\forall\;x)(\exists\;y)(\forall\;t)(t\in y\iff t\in x\land \varphi(t))$
\end{center}\bigskip
\begin{center}\large
    {\color{def}FULL VERSION:} niech $\varphi(t, z_0, ..., z_n)$ będzie formułą jezyka teorii mnogści. Wtedy pozostałe zmienne wolne będa parametrami (zapis skrócony $z_0, ..., z_n:= \overline z$)\smallskip\\
    Dla każdego układu parametrów i dla każdego $x$ istnieje $y$ taki, że dla każdego $t\in y$ $t$ należy do $x$ i $t$ spełnia formułę $\varphi$\smallskip\\
    $(\forall\;z_0)...(\forall\;z_n)(\forall\;x)(\exists\;y)(\forall\;t)(t\in y\iff t\in x\land \varphi(t, z_0, ..., z_n))$
\end{center}\bigskip
Weźmy półprostą otwartą:
$$(0, +\infty)=\{x\in\R\;:\;x>0\},$$
druga półprosta to
$$(1, +\infty)=\{x\in\R\;:\;x>1\}$$
i tak dalej. Czyli ogólna definicja półprostej to:
$$(a, +\infty)=\{x\in \R\;:\;x>a\}.$$
Dla każdej z tych półprostych trzeba wziąc inną formułę, które wszystkie są zdefinio-\\wane za pomocą formuły
$$\varphi(x, a)=(x>a),$$
gdzie $a$ funkcjonuje jako parametr.
\subsection{AKSJOMAT ZASTĘPOWANIA}
Ostatni aksjomat konstrukcyjny, jest to schemat rodziny aksjomatów\smallskip\\
\begin{center}\large
    {\color{def}SIMPLIFIED VERSION:} niech $\varphi(x, y)$ będzie formułą języka teorii mnogości taką, że:\smallskip\\
    $(\forall\;x)(\exists\;!\;y)\varphi(x, y).$\smallskip\\
    Wówczas dla każdego zbioru $x$ istnieje zbiór $\{z\;:\;(\exists\;t\in x)\;\varphi(t, z)\}$\smallskip\\
    $(\forall\;x)(\exists\;y)(\forall\;z)\;(z\in y\iff (\exists\;t\in x)\;\varphi(t, z))$
\end{center}\medskip
Czyli każdy zbiór można \emph{\color{acc}opisać za pomocą operacji}.\bigskip\\
\begin{center}\large
    {\color{def}FULL VERSION:} niech $\varphi(x, y, p_0, ..., p_n)$ będzie formułą języka teorii mnogości. \smallskip\\
    $(\forall\;p_0), ..., (\forall\;p_n)\;((\forall\;x)\;(\exists\;!y)\;\varphi(x, y, \overline p)\implies (\forall\;x)(\exists\;y)(\forall\;z)\;(z\in y\iff (\exists\;t\in x)\;\varphi(t, z, \overline p)))$
\end{center}

\subsection{KONSTRUKCJE NA ZBIORACH SKOŃCZONYCH}
Niech $x, y$ będą dowolnymi zbiorami. Wtedy definiujemy:\medskip\\
    \indent $x\cap y=\{t\in x\;:\;t\in y\}$\smallskip\\
    \indent $x\setminus y=\{t\in x\;:\; t\notin y\}$\smallskip\\
    \indent $x\times y=\{z\in \Po{\Po{x\cup y}}\;:\;(\exists\; s\in x)(\exists\;t\in y)\;z=\parl s, t\parr\}$\medskip\\
Formalnie stara definicja iloczynu kartezjańskiego nie działa w nowych warunkach, bo \\nie wiemy z czego wyróżnić tę parę uporządkowaną. Ponieważ $s, t\in x\cup y$, mamy
$$\{s\}, \{s, t\}\subseteq x\cup y,$$
a więc 
$$\{\{s\}, \{s, t\}\}\subseteq \Po{x\cup y}.$$
Czyli nasza para uporządkowana jest elementem zbioru potęgowego zbioru potęgowego sumy zbiorów.\medskip\\
    \indent $\bigcap x=\{z\in \bigcup x\;:\;(\forall\;y\in x)\;z\in y\}$ i wówczas $\bigcap\emptyset=\emptyset$\bigskip\\
\podz{gr}\bigskip\\
{\color{def}RELACJA} - definiujemy $\rel r$ jako dowolny zbiór par uporządkowanych:
$$\rel r :=(\exists\;x)(\exists\;y)\;r\subseteq x\times y$$
{\color{def}FUNKCJA} - relcja, która nie ma dwóch par o tym samym poprzedniku i różnych następni-\\kach:
$$\funk f := \rel f \land (\forall\;x)(\forall\;y)(\forall\;z)\;(\parl x,y\parr\in f\land \parl x, z\parr \in f)\implies y = x$$
Dziedzinę i zbiór wartości możemy wówczas zdefiniować jako:
$$\dom f = \{x\in \bigcup \bigcup f\;:\;(\exists\;y)\parl x,y\parr \in f\}$$
$$\rng f = \{y\in \bigcup \bigcup f\;:\;(\exists\;x)\parl x,y\parr \in f\},$$
ponieważ 
$$\{\{x\}, \{x, y\}\}\in f\implies \{x\}, \{x, y\}\in \bigcup f\implies x,y\in\bigcup\bigcup f$$
\emph{Dopóki działamy na zbiorach skończonych, wynikiem operacji zawsze będzie kolejny zbiór skończony - niemożliwe jest otrzymanie zbioru nieskończonego.}

\subsection{AKSJOMAT NIESKOŃCZONOŚCI}
\begin{center}\large
    Istnieje {\color{emp}zbiór induktywny}:\smallskip\\
    $(\exists\;x)\;(\emptyset\in x\land (\forall\;y\in x)\;(y\cup\{y\}\in x))$
\end{center}
Na początku do naszego zbioru $x$ dodajemy $\emptyset$. Potem, skoro $\emptyset$ należy do $x$, to należy też \\$\{\emptyset\}$. Ale skoro do $x$ należy $\emptyset\cup\{\emptyset\}$, to również $\{\emptyset\cup\{\emptyset\}\}$ jest jego elementem i tak dalej.\bigskip\\
\podz{def}\bigskip
\begin{center}\large
    {\color{def}TW.} Istnieje zbiór induktywny najmniejszy względem zawierania, czyli taki, który zawiera się w każdym innym zbiorze induktywnym.
\end{center}\bigskip
\dowod
Niech $x$ będzie zbiorem induktywnym, który istnieje z aksjomatu nieskończoności. Niech
$$\omega=\bigcap\{y\in\Po x\;:\;y \texttt{ jest zbiorem induktywnym}\}$$
Chcę pokazać, że $\omega$ jest zbiorem induktywnym, czyli $\emptyset\in\omega$.
$$\emptyset\in\omega\iff\emptyset\in y \texttt{ dla każdego zbioru induktywnego }y\subseteq x$$
Ponieważ każdy zbiór induktywny zawiera $\emptyset$, także $\omega$ zawiera $\emptyset$.\medskip\\
Pozostaje pokazać, że dla dowolnego $t\in\omega$ mamy
$$t\cup\{t\}\in \omega$$
Dla każdego zbioru induktywnego $y\subseteq x$ mamy $t\in y$. ale ponieważ $y$ jest zbiorem induktyw-\\nym, mamy 
$$t\cup\{t\}\in y.$$
Z definicji przekroju zbioru $x$ mamy
$$t\cup\{t\}\in \bigcap \{y\in \Po x\;:\;\texttt{ y jest zbiorem induktywnym}\}=\omega$$
Czyli istnieje zbiór induktywny $\omega$ będący przekrojem wszystkich innych zbiorów induktyw-\\nych. Pokażemy teraz, że jest to zbiór najmniejszy.\medskip\\
Niech $z$ będzie dowolnym zbiorem induktywnym. Wtedy $z\cap x$ jest zbiorem induktywnym i \\$z\cap x\subseteq x$. Czyli $z$ jest jednym z elementów rodziny, której przekrój daje $\omega$:
$$z\cap x\supseteq \{y\in\Po x\;:\; Y\texttt{ zb. ind.}\}=\omega$$
\kondow
\podz{gr}\bigskip\\
Każdy element $\emptyset,\;\{\emptyset\},\;\{\emptyset,\{\emptyset\}\}...$ możemy utoższamić z {\color{acc}kolejnymi liczbami naturalnymi}. W ta-\\kim razie ten najmniejszy zbiór induktywny będzie utożsamiany ze zbiorem liczb natural-\\nych. Konsekwencją tego jest \emph{\color{emp}zasada indukcji matematycznej}.\smallskip\\
Niech $\varphi(x)$ będzie formułą ozakresiie zmiennej $x\in\N$ takiej, że zachodzi $\varphi(0)$ oraz
$$(\forall\;n\in\N)\;\varphi(n)\implies\varphi(n+1).$$
Wówczas 
$$(\forall\;z\in\N)\;\varphi(n)$$
\dowod
Niech 
$$A=\{n\in\N\;:\;\varphi(n)\}.$$
Wtedy $A\in\N$ oraz $A$ jest induktywny. Kolejne zbiory należące do zbioru induktywnego \\utożsamialiśmy z $n\in\N$, więc skoro $\varphi(n)$ należy do tego zbioru induktywnego, to również \\$\varphi(n+1)$ należy do $A$. Skoro $A$ jest zbiorem induktywnym, to $\N\subseteq A$, więc $A=\N$.
\kondow

\subsection{AKSJOMAT REGULARNOŚCI}
Do tej pory poznaliśmy aksjomaty o instnieniu i serie aksjomatów konstrukcyjnych. Ak-\\sjomat regularności nie jest żadnym z nich.\bigskip
\begin{center}\large
    W każdym niepustym zbiorze istnieje element {\color{emp}$\in-$minimalny:}\smallskip\\
    $\color{acc}(\forall\;x)\;x\neq\emptyset\implies ((\exists\;y\in x)\;(\forall\;z\in x)\;\neg\;z\in y)$,\medskip\\
    a więc eliminowane są patologie jak np: $x\in x$, $y\in y\in x$.
\end{center}\bigskip

Antynomia Russlla,
$$\{x\;:\;x\notin x\},$$
jest eliminowana przez aksjomat regularności.

\subsection{AKSJOMAT WYBORU}
\begin{center}\large
    Dla każdej {\color{emp}rozłącznej rodziny parami rozłącznych \\zbiorów} niepustych {\color{def}istnieje SELEKTOR}\smallskip\\
    $(\forall\;x)\;{\color{acc}(}(\forall\;y,z\in x)\;{\color{tit}(}y\neq\emptyset\;\land\;(y\neq z\implies y\cap z=\emptyset){\color{tit})}\implies(\exists\;s)(\forall\;y\in x)(\exists\;!t)\;t\in s\cap y{\color{acc})}$
\end{center}\bigskip

Problematyczne nie jest znalezienie punktów, które są reprezentantami zbiorów naszej \\rodziny, a wskazanie zbioru, który je wszystkie zawiera. Dlatego w tym może nam pomóc akjomat wyboru. Wystarczy pokazać, że rozważamy rodzinę rozłącznych zbiorów i już z \\tego wiemy, że możemy wybrać selektor. Handy.\bigskip\\
{\large\color{emp}PARADOKS BANACHA-TARSKIEGO:}\medskip\\
Kulę możemy rozłożyć na {\color{acc}5 kawałków} i przesuwać je izometrycznie w taki sposób, żeby \\złożyć z nich {\color{acc}dwie identyczne kule} jak ta, którą mieliśmy na początku. Kawałki na \\które dzielimy są niemieżalne, nie mają objętości, są maksymalnie patologiczne, ale \\nadal możemy powiedzieć że istnieją korzystając z aksjomatu wyboru. Daje on nam tylko informację, że {\color{acc}istnieje selektor, a nie o tym jak on wygląda,} więc może być absurdalny i patologiczny jak tylko ma ochotę.\bigskip\\
\podz{tit}\bigskip\\
\begin{center}\large
    {\color{def}FUNKCJA WYBORU} - niech $\rodz A$ będzie rodziną zbiorów \\niepustych. Funkcją wyboru dla rodziny $\rodz A$ nazywamy \\wtedy {\color{acc}dowolną funkcję $f$:}\smallskip\\
    $f:\rodz A\to\bigcup \rodz A$\smallskip\\
    $(\forall\;A\in\rodz A)\;f(A)\in A$\medskip\\
    \normalsize Aksjomat wyboru jest równoważny temu, że dla każdej rozłącznej rodziny niepustych zbiorów istnieje funkcja wyboru (selektor).
\end{center}

\begin{center}\large
    Dla dowolnych dwóch zbiorów $A$, $B$ zachodzi\smallskip\\
    $|A|\leq|B|\lor|B|\leq|A|$
\end{center}\bigskip
\dowod

Musimy skonstruować zbiór częściowo uporządkowany $X$, do którego będziemy mogli zastosować LKZ. Elementami tego zbioru niech będą przybliżenia tego, co chcemy otrzymać:
$$X=\{f\;:\;fnc(f)\;\land\;dom(f)\subseteq A\;\land\;rng(f)\subseteq B\;\land\;f\;jest\;1-1\}$$