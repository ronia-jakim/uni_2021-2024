\section{LICZBY PORZĄDKOWE}
\subsection{LEMAT KURATOWSKIEGO-ZORNA}
\begin{center}\large
    {\color{def}LEMAT KURATOWSKIEGO-ZORNA:}\medskip\\
    Jeśli $\parl X,\leq\parr$ jest zbiorem częściowo uporządkowanym, w którym {\color{acc}każdy łańcuch jest ograniczony z góry}, to w $X$ istnieje {\color{emp}element maksymalny}.
\end{center}\bigskip
\podz{gr}\bigskip

\begin{center}\large
    Suma przeliczalnie wielu przeliczalnych zbiorów jest przeliczalna:\smallskip\\
    $\quad(\forall\;n\in\N)\;|A_n|\leq\aleph{0}\implies \aleph_0\geq\bigcup\limits_{n\in\N}A_n$
\end{center}\bigskip

\dowod
Ponieważ $|A_n|\leq \aleph_0$, to istnieje bijekcja
$$f_n:\N\to A_n.$$
Chcemy pokazać, że istnieje też bijekcja:
$$f:\N\times\N\to\bigcup\limits_{n\in\N}A_n$$
$$f(n,k)=f_n(k)\quad(\kawa)$$

Musimy znać wszystkie elementy $(f_n)$ jednocześnie, więc skorzystamy z aksjomatu wyboru. \\Rozpatrzmy zbiór funkcji:
$$F_n=\{\varphi\in S^\N_n\;:\;\varphi\texttt{ jest bijekcją}\}$$
dla $n\in\N$, gdzie $S_n^\N$ oznacza wszstkie funkcje
$$g:\N\to A_n$$
Niech $F$ będzie funkcją wyboru dla rodziny 
$$\{F_n\;:\;n\in\N\},$$ 
czyli każdej rodzinie przypisujemy element tej rodziny:
$$F(F_n)\in F_n.$$
Opiszmy $(\kawa)$ korzystając z funkcji wyboru:
$$f(n, k)=F(F_n)(k).$$
Ponieważ $F(F_n)$ jest bijekcją, to również funkcja $f$ jest bijekcją.
\kondow

\podz{gr}\bigskip

\begin{center}\large
    Dla dowolnych zbiorów $A$, $B$ zachodzi\smallskip\\
    $|A|\leq|B|\;\lor\;|B|\leq|A|$
\end{center}
\dowod
Musimy skonstruować zbiór częściowo uporządkowany $X$, do którego będziemy mogli zasto-\\sować LKZ. Chcemy pokazać, że istnieje iniekcja lub suriekcja między tymi dwoma zbiora-\\mi, więc potrzebujemy zbioru zawierającego funkcje z jednego do drugiego:
$$X=\{f\;:\;fnc(f)\;\land\;dom(f)\subseteq A\;\land\;rng(f)\subseteq B\;\land\; f\;jest\;1-1\}.$$

Rozpatrzmy porządek $\parl X, \subseteq$. Aby zastosować do niego LKZ musimy sprawdzić założenia. \\Weźmy łańcuch $X$:
$$\rodz L\subseteq X.$$
Musimy pokazać, że ma on ograniczenie górne. Niech
$$L=\bigcup \rodz L.$$
Ponieważ każdy element $\rodz L\in L$, to $L$ jest ograniczeniem górnym $\rodz L$.\medskip\\
Należy teraz pokazać, że $L$ jest elementem zbioru $X$, czyli spełnia warunki:\smallskip\\
\indent 1. {\color{emp}$L$ jest zbiorem par uporządkowanych} - bezpośrednio z tego, że $L$ jest sumą łań-\\cucha $\rodz L\subseteq X$.\smallskip\\
\indent 2. {\color{emp}$L$ jest funckją,} czyli
$$(\forall\;x,y,z)\;(\parl x, z\parr\in L\;\land\;\parl x, z\parr\in L)\implies y=z.$$
Ustalmy dowolne $x, y, z$ takie, że $\parl x,y\parr\in L$ oraz $\parl x, z\parr\in L$. Zatem istnieją $F, G\in \rodz L$ takie, że
$$\parl x, y\parr\in F\land\parl x, z\parr\in G.$$
Ponieważ $\rodz L$ ma {\color{acc}ograniczenie górne i jest łańcuchem}, to wszystkie jego elementy mogą być między sobą porównywane. Bez straty ogólności możemy więc założyć, że $\color{acc}F\subseteq G$ i z tego wynika, że
$$(\parl x, y\parr\in G\;i\;\parl x, z\parr\in G)\implies y=z,$$
bo $fnc(G)$.\smallskip\\
\indent 3. {\color{emp}$dom(L)\subseteq A$} z tego, że $\rodz L\subseteq X$.\smallskip\\
\indent 4. {\color{emp}$rng(L)\subseteq A$} z tego, że $\rodz L\subseteq X$.\smallskip\\
\indent 5. {\color{emp}$L$ jest funkcją różnowartościową,} czyli $\parl x, y\parr=\parl z, y\parr\implies x=z$.\smallskip\\
Ustalmy dowolne $x, y, z$ takie, że 
$$\parl x,y\parr\in L \; i \; \parl z,y\parr\in L.$$
Zatem istnieją $F, G\in \rodz L$ takie, że
$$\parl x,y\parr\in F\;\land\;\parl z,y\parr\in G.$$
Ponieważ $\rodz L$ jest łańcuchem, to możemy założyć, że $F\subseteq G$, a ponieważ $\rodz L\subseteq X$ i $X$ zawiera jedynie iniekcje, to
$$\parl x,y\parr\in G\;\land\;\parl z,y\parr\in G\implies x=z.$$
Ponieważ pokazaliśmy, że {\color{acc}dowolny łańcuch $X$ jerst ograniczony z góry, to na mocy LKZ \\w $X$ istnieje element maksymalny}
$$\varphi\in X.$$
Rozpatrzmy trzy możliwości:
\indent 1. {\color{emp}$dom(\varphi)=A$:} wówczas z definicji zbioru $X$ otrzymujemy 
$$\varphi : A\xrightarrow[na]{1-1} B$$
a więc $|A|\leq|B|$.\smallskip\\
\indent 2. {\color{emp}$rng(\varphi)=B$:} wtedy $|B|\leq |A|$, bo
$$\varphi\;:\;dom(\varphi)\xrightarrow[na]{1-1} B$$
$$\varphi^{-1}\;:\;B\xrightarrow[na]{1-1} dom(\varphi)\subseteq A.$$
\indent 3. {\color{emp}$dom(\varphi)\neq A\;\land\;rng(\varphi)\neq B$:} czyli $dom(\varphi)\subsetneq A$ i $rng(\varphi)\subsetneq B$, zatem istnieją $s\in A\setminus dom(\varphi)$ oraz $t\in B\setminus rng(\varphi)$. W takim razie $\varphi$ może być rozszerzona do:
$$\varphi'=\varphi\cup\{\parl s,t\parr\}.$$
$$\varphi'\in X$$ nie jest iniekcją, bo $t\notin rng(\varphi)$. Dodatkowo,
$$\varphi\subsetneq \varphi',$$
czyli $\varphi$ nie jest elementem maksymalnym w $X$, stąd {\color{emp}zachodzi tylko 1 lub 2}, czyli $|A|\leq |B|$ lub $|B|\leq |A|$.
\kondow

\subsection{DOBRE PORZĄDKI}
{\color{acc}Dobry porządek }- w każdym niepustym podzbiorze $\parl X, \leq\parr$ istnieje element najmniejszy.
\begin{center}\large
    {\color{def}CZĘŚCIOWY LINIOWY DOBRY PORZĄDEK} $\parl X,\leq\parr,\;Lin(X)$???\smallskip\\
    $(\forall\;A\subseteq X)\;A\neq\emptyset\implies ((\exists\;a\in A)(\forall\;x\in A)\;x\leq A)$\smallskip\\
    $(\forall\;a,b\in A)\;a\leq b\;\lor\;b\leq a$\smallskip\\
    oraz $\leq$ jest zwrotny, przechodni i słabo antysymetryczny.
\end{center}\bigskip
Do tej pory ostry porządek < definiowaliśmy jako skrót
$$x<y\iff x\leq y\land x\neq y.$$
Teraz chcemy, żeby stał się on bytem. Seria twierdzeń z tym związanych:\medskip
\begin{itemize}
    \item relacja < jest przechodznia i silnie antysymetryczne
    \item jeśli < jest relacją przechodnią i silnie antysymetryczną, to relacja zadana warun-\\kiem $x\leq y \iff x< y\lor x=y$ jest częściowym porządkiem
    \item każdemu częściowemu porządkowi odpowiada tylko jeden osry porządek i każdemu os-\\tremu porządkowi odpowiada tylko jeden częściowy porządek.
\end{itemize}
\begin{center}
    {\color{def}SPÓJNOŚĆ} to warunek mówiący, że\smallskip\\
    $(\forall\;x,y)\;x\neq y\implies (xRy\;\lor\; yRx)$
\end{center}\bigskip
\podz{gr}\bigskip\\
{\color{acc}PRZYKŁADY} - dobry porządek\medskip\\
\indent 1. $\parl\N,\leq\parr$ 0 zasada minimum mówi, że w każdym niepustym podzbiorze $\N$ istnieje element najmnijszy, co jest róownoważne zasadzie indukcji matematycznej\smallskip\\
\indent 2. $\parl \{1-\frac1{n+1}\;:\;n\in\N\}, \leq\parr$ - izomorficzne ze zbiorem $\N$\smallskip\\
\indent 3. $\parl \{1-\frac 1{n+1}\}\cup\{1\}, \leq\parr$\smallskip\\
\indent 4. $\parl \{1-\frac 1{n+1}\}\cup\{2-\frac 1{n+1}\}, \leq\parr$\smallskip\\
\indent 5. $\parl n-\frac1m\;:\;n,m\in\N,\leq\parr$\bigskip\\
\podz{gr}\bigskip
\begin{center}\large
    {\color{def}ODCINEK POCZĄTKOWY} - niech $\parl X,\leq\parr$ będzie zbiorem \\z dobrym porządkiem $\leq$ i $a\in X$. Wówczas \\odcinkiem początkowym tego zbioru wyznaczonym \\przez $a$ jest zbiór\smallskip\\
    $pred(X, a,\leq)=\{x\in X\;:\;x<a\}$
\end{center}
W przykładach wyżej każdy zbiór jest odcinkiem początkowym dla zbioru następnego. \\'Krótsze porządki' są odcinkami początkowymi dla dłuższych porządków.\bigskip\\
{\large\color{acc}TWIERDZENIE:} dla dowolnego $a\in X$
$$pred(X, a, \leq)\not\simeq X$$
\dowod
Przypuśćmy, nie wprost, że dla pewnego $a\in X$ mamy
$$pred(X, a,\leq)\simeq X,$$
czyli isitnieje izomorfizm
$$f:X\to pred(X, a,\leq).$$
Wtedy $f(a)<a$, bo izomorfizm zachowuje porządek, i zbiór
$$A=\{x\in X\;:\;f(x)<x\}$$
jest niepusty. Niech $b=\min A$, ale wtedy
$$f(b)<b\implies f(f(b))<f(b),$$
czyli $b>f(b)\in A$, co jest sprzeczne z $b=\min A$.
\kondow
\podz{gr}\bigskip\\
Niech $\parl X, \leq_X\parr,\;\parl Y,\leq_Y\parr$ będą zbiorami dobrze uporządkowanymi. Wtedy zachodzi jedna z trzech możliwości:\smallskip\\
\indent 1. te dwa zbiory są {\color{acc}izomorficzne} $(X\simeq Y)$, czyli są tej samej długości\smallskip\\
\indent pierwszy jest dłuższy od drugiego:
$$(\exists\;a\in X)\;\parl pred(X, a, \leq_X), \leq\parr\simeq \parl Y,\leq_Y\parr$$
\indent 3. drugi jest dłuższy od pierwszego:
$$(\exists\;a\in Y)\;\parl pred (Y, a, \leq_Y), \leq\parr\simeq \parl X, \leq_X\parr$$
\emph{Wypadałoby to wszystko udowodnić, ale to jest przyjemny wykład i uznamy, że wszystko \\śmiga, żeby przejść do bardziej podniecających rzeczy, gdzie będziemy korzystać z po-\\prawności tego nieistniejącego dowodu :3}
\subsection{ZBIÓR TRANZYTYWNY}
\begin{center}
    \emph{Elementy moich elemntów są moimi elementami!}\medskip\\
    Zbiór $A$ nazywamy zbiorem {\color{acc}TRANZYTYWNYM}, gdy każdy jego element jest zarazem jego podzbiorem:\smallskip\\
    $(\forall\;x\in A)\;x\subseteq A$
\end{center}
$\emptyset$ jest zbiorem tranzytywym, bo nie ma elementów - ponieważ nie istnieją, to mogą mieć \\dowolne własności, w szczególnośći mogą być podzbiorami $\emptyset$. Tak jak wwierszy \emph{Na wyspach Bergamota}.\medskip\\
$\{\emptyset\}$ - jego jedyny element to zbiór pusty, który jest jednocześnie jego podzbiorem.\medskip\\
$Tran(\omega)$ - każda liczba naturalna jest zbiorem liczb od siebie mniejszych - dowód na \\liście zadanek :v\bigskip\\
\podz{gr}\bigskip\\
Jeżeli zbiór jest tranzytywny, to tranzytywna jest też jego {\color{def}suma, zbiór potęgowy i jego następnik:}
$$Tran(A)\implies Tran(\bigcup A)\implies Tran(\Po A)\implies Tran(A\cup \{A\})$$
\dowod
Udowodnimy, że $Tran(A)\implies Tran(A\cup \{A\})$\medskip\\
Ustalmy dowolne $x\in A\cup\{A\}$. Wtedy zachodzi jeden z dwóch przypadków:\medskip\\
\indent 1. $x\in A$, a ponieważ $Tran(A)$, to
$$(\forall\;y\in x)\;y\in A$$
\indent 2. $x\in\{A\}$, czyli $x=A$, a więc z $Tran(A)$ otrzymujemy, że $y\in x\implies y\in A\implies y\in \{A\}$.
\kondow
\subsection{LICZBY PORZĄDKOWE}
\begin{center}\large
    Zbiór tranzytywny $A$ nazywamy {\color{def}LICZBĄ PORZĄDKOWĄ}, \\jeśli spełnia warunek\smallskip\\
    $(\forall\;x,y\in A)\;x\in y\;\lor\;x=y\;\lor\; y\in x$\smallskip\\
    i używamy oznaczenia $On(A)$.
\end{center}
Jeśli $On(\alpha)$, to $\alpha$ jest dobrze uporządkowane przez $\in$, czyli każdy niepusty zbiór $A\subseteq \alpha$ ma element $\in$-minimalny:
$$(\forall\;A\subseteq \alpha)\;A\neq\emptyset\implies(\exists\;x\in A)(\forall\;y\in A)\;x=y\;\lor\;x\in y,$$
co wynika z aksjomatu regularności.\bigskip\\
\podz{gr}\bigskip\\

{\large\color{def}PODSTAWOWE WŁASNOŚCI LICZB PORZĄDKOWYCH:}\medskip\\
$\alpha, \beta$ - liczby porządkowe, $C$- zbiór liczb porządkowych\medskip\\
\indent {\color{acc}1. $(\forall\;x\in\alpha)\;On(\alpha)$} - elementy liczby porządkowej są liczbami porzadkowymi.\smallskip\\
Ustalmy dowolne $x\in\alpha.$ Ponieważ $Tran(\alpha)$, to
$$x\in\alpha.$$
Zatem $Lin(x)$, bo $Lin(\alpha)$. Ustalmy dowolne $y\in x$ i $x\in y$. Skoro $x\subseteq \alpha$, to $y\in\alpha$, czyli $y\subseteq \alpha$, zatem $z\in \alpha$. W takim razie $x, z$ są porównywalne jako elementy $\alpha$. Mamy trzy możliwości: $z\in x$, $x\in z$ (sprzeczne z aksj. regularności), $z=x$ (sprzeczne z aksj. regularności).\medskip\\
\indent {\color{acc}2. $\alpha\in \beta\iff\alpha\subset \beta$}\medskip\\
\indent {\color{acc}3. $\alpha\in \beta\;\lor\;\alpha=\beta\;\lor\;\beta\in\alpha$} - dowolne dwie liczby porządkowe są porównywalne.\smallskip\\
Niech $A=\alpha\cap\beta$. Wtedy $On(A)$. Przypuśćmy, że
$$A\neq \alpha\;\land\;A\neq\beta.$$
Wówczas $A$ jest prawdziwym podzbiorem zarówno $\alpha$ jak i $\beta$. Ale z 2 mamy
$$A\in \alpha\;\land\;A\in\beta,$$
czyli
$$A\in \alpha\cap\beta=A.$$
Jest to sprzeczne z aksjomatem regularności, więc $A=\alpha$ lub $A=\beta$, czyli $\alpha\subseteq \beta$ lub \\$\beta\subseteq\alpha$, co z 2 daje nam $\alpha\in \beta$ lub $\beta\in\alpha$.\medskip\\
\indent {\color{acc}4. $Tran(C)\implies On(C)$}\medskip\\
\indent {\color{acc}5. $C\neq\emptyset\implies(\exists\;\alpha\in C)(\forall\;\beta\in C)\;\alpha=\beta\;\lor\;\alpha\in\beta$.}\bigskip\\

Liczbę porządkową $\alpha$ utożsamiamy ze zbiorem dobrze uporządkowanym $\color{def}\parl \alpha, \in\parr$. Możemy w takim razie mówić o $\color{def}pred(\alpha,\in,\beta)$, ale skrócimy to do zapisu:
$$\color{acc}pred(\alpha,\in,\beta)=pred(\alpha,\beta)=\{x\in\alpha\;:\;x\in\beta\}=\beta,$$
czyli każda liczba porządkowa jest zbiorem liczb porządkowych od niej mniejszych.\bigskip\\
Jeśli $On(\alpha)$, to wtedy $\alpha\cup\{\alpha\}$ jest najmniejszą liczbą porządkową większą od $\alpha$ i nazywamy ją {\color{def}NASTĘPNIKIEM} porządkowym liczby $\alpha$
$$\alpha\cup\{\alpha\}:=\alpha+1$$
\begin{center}\large
    Nie istnieje zbiór wszystkich liczb porządkowych\medskip\\
    \normalsize\emph{paradoks Burali-Forti}
\end{center}
\dowod
Przypuścmy nie wprost, że $\color{acc}ON$ jest zbiorem wszystkich liczb porządkowych. Wtedy 
$$Tran(ON),$$ 
bo jeśli $\alpha\in ON$ i $\beta\in \alpha$, to $\beta\in ON$. Ponadto, $Lin(ON)$ z własności 3. Zatem 
$$On(ON),$$ 
czyli $ON\in ON$, co jest sprzeczne z aksjomatem regularności.
\kondow

\podz{def}\bigskip
\begin{center}\large
    Nich $\parl X,<\parr$ będzie zbiorem dobrze \\uporządkowanym. Wtedy istnieje dokładnie \\{\color{def}jedna liczba porządkowa $\alpha$ taka, że\smallskip\\
    $\parl X,<\parr\simeq \parl\alpha,\in\parr$}
\end{center}
Czyli każdy zbiór dobrze uporządkowany jest izomorficzny z jakąś liczbą porządkową.\bigskip\\
\dowod
{\large\color{def}1. JEDYNOŚĆ}\medskip\\
Przypuśćmy, nie wprost, że istnieją dwie różne liczby porządkowe $\alpha,\;\beta$ spełniające zależ-\\ność z twierdzenia. Wtedy
$$\alpha\simeq\beta,$$
co jest sprzeczne z ich różnością - któraś musi być mniejsza i wyznaczać odcinek począt-\\kowy w drugiej. Zbiór nie może być izomorficzny ze swoim odcinkiem poczatkowym.\bigskip\\
{\large\color{def}2. ISTNIENIE}\medskip\\
Zdefiniujmy zbiór
$$Y=\{a\in X\;:\;(\exists\;!\gamma_a)\;On(\gamma_a)\;\land\;\parl pred(X,a,<), <\parr\simeq\gamma_a\},$$
czyli wybieram podzbiory $X$, dla których twierdzenie zachodzi. Zauważmy, że $Y\neq \emptyset$, bo w $X$ istnieje element minimalny (z dobrego porządku). \medskip\\
Dla $a\in Y$ rozważmy izomorfizm
$$\varphi_a:pred(X,a,<)\to \gamma_a.$$
Niech $b\in Y$ o $b<a$. Wtedy
$$\varphi_a(b)\in\gamma_a$$
\pmazidlo
\draw[gr, very thick] (0, 0)--(5, 0);
\draw[gr, very thick] (0,2)--(5,2);
\draw[emp, ultra thick, ->] (2.5, 2)--(2.5, 0);
\node at (5.3, 2) {a};
\node at (5.3, 0) {$\gamma_a$};
\node at (2.5, 2.3) {b};
\node at (2.5, -0.4) {$\varphi_a(b)$};
\kmazidlo

W takim razie, $\varphi_{a \obet pred(X, b, <)}$ jest izomorfizmem pomiędzy $pred(X, b, <)$ i $\varphi_a(b)$. W takim razie $b\in Y,$ czyli $Y$ jest zamknięty w dół.\medskip\\
Stąd możemy wnioskować, że $X=Y$ lub $Y=pred(X, c, <)$. Załóżmy, że $Y=pred(X, c, <):$
$$X\neq Y\implies X\setminus Y\neq\emptyset.$$
Niech $c=\min(X\setminus Y)$, wówczas 
$$Y=pred(X, c, <).$$
Mam węc zbiór $Y$, z którego każdym elementem jest związana jakaś liczba porządkowa. \\Z aksjomatu zastępowania mogę stworzyć zbiór wszystkich tych liczb porządkowych.
$$f:Y\to ON$$
$$f(a)=\gamma_a$$
$$A=rng(f)=\{\gamma_a\;:\;a\in Y\}.$$
Wystarczy pokazać:\medskip\\
\indent 1. $Tran(A)\implies On(A)$ (z 4.):\smallskip\\
Ustalmy $\xi\in A$ oraz $\zeta\in \xi$. Skoro $\xi\in A$, to $\xi=\gamma_a$ dla pewnego $a\in Y$. Wtedy istnieje $b<a$ takie, że $\varphi_a(b)=\zeta$. Stąd wynika, że $\zeta=\gamma_b$, czyli $\zeta\in A$.\medskip\\
\indent 2. $f$ jest izomorfizmem porządkowym.\smallskip\\
Jest funkcją 1-11 z definicji zbioru $Y$, a funkcją "na" z definicji zbioru $A$. Zachowuje \\porządek, bo rozważamy odcinki początkowe.\medskip\\
\indent 3. $X=Y$ \smallskip\\
$Y=pred(X, c, <)$, a {\color{emp}pokazaliśmy, że $c\in Y$, bo $Y\simeq On(\alpha)$, więc jest dobrym porządkiem (ma elememnt najmniejszy)}. W takim razie tu byłaby sprzeczność.\\
Wyżej zakładaliśmy, że $X\neq Y\implies Y=pred(X, c, <)$. Ponieważ !?!?!?!?!?
\kondow
{\color{dygresyja}TWIERDZENIE NA BOCZKU}\smallskip\\
{\color{acc}TWIERDZENIE HARTOGSA} - Dla każdego zioru $X$ istnieje liczba porządkowa $\alpha$, dla której \\nie istnieje funkcja różnowartościowa w zbiór $X$\bigskip\\
\podz{def}\bigskip
\begin{center}\large
    {\color{def}TYPEM PORZDKOWYM} zbioru dobrze uporządkowanego \\nazywamy liczbę porządkową, z którą jest \\on homeomorficzny.\smallskip\\
    $ot(\N,\leq)=ot(\parl\{1-\frac1{n+1}\;:\;n\in\N\}\parr,\leq)=\omega$\smallskip\\
    $ot (\parl\{1-\frac1{n+1}\;:\;n\in\N\}\cup\{1\},\leq\parr)=\omega+1$
\end{center}

\subsection{DZIAŁANIA NA LICZBACH PORZĄDKOWYCH}
Niech $\alpha,\;\beta$ będą liczbami porządkowymi. Wówczas {\color{def}dodawanie definiujemy}:
{\large$$\alpha+\beta=ot(\alpha\times\{0\}\cup\beta\times\{1\}, \leq)$$}
\pmazidlo
\draw[acc, ultra thick] (0, 0) -- (2, 0);
\node at (1, -0.3) {$\color{acc}\alpha$};
\draw[def, ultra thick] (1.8, 1.3) -- (3.8, 1.3);
\node at (2.8, 1) {$\color{def}\beta$};
\node at (-0.3, 0) {0};
\node at (1.5, 1.3) {1};
\draw[gr, very thick] (-0.5, 1.8) rectangle (4.1, -0.5);
\node at (3.8, -0.8) {$\color{gr}\alpha+\beta$};
\kmazidlo
czyli najpierw rozdzielamy je, a potem sumujemy. Relację porządku na sumie liczb po-\\rządkowych definiujemy (porządek leksykograficzny):
{\large$$\parl\gamma, i\parr\leq_{lex}\parl\xi, j\parr\iff i<j\;\lor\; (i=j\;\land\;\gamma<\xi).$$}
{\color{def}Mnożenie liczb porządkowych} to z kolei typ porządkowy ich iloczynu z porządkiem leksy-\\kograficznym:
{\large$$\alpha\cdot\beta=ot(\beta\times\alpha,\leq_{lex})$$}
\pmazidlo
\draw[acc, ultra thick] (0, 0)--(0, 1.5);
\draw[acc, ultra thick] (0.4, 0)--(0.4, 1.5);
\draw[acc, ultra thick] (0.8, 0)--(0.8, 1.5);
\draw[acc, ultra thick] (1.2, 0)--(1.2, 1.5);
\draw[acc, ultra thick] (1.6, 0)--(1.6, 1.5);
\draw[acc, ultra thick] (2, 0)--(2, 1.5);
\draw[def, ultra thick] (0, 0)--(2, 0);
\node at (1, -0.4) {$\color{def}\beta$};
\node at (-0.3, 0.7) {$\color{acc}\alpha$};
\draw[gr, very thick] (-0.7, 1.8) rectangle (2.3, -0.6);
\node at (2, -0.9) {$\color{gr}\alpha\cdot\beta$};
\kmazidlo
czyli bierzemy $\beta$ kopii $\alpha$ - wygodniej na to patrzeć jak na takiego jerzyka z iloczynu \\kartezjańskiego.\medskip\\
Kilka przykładów:\smallskip\\
\indent $\omega+\omega=ot(\{1-\frac1{n+1}\;:\;n\in\N\}\cup\{2-\frac1{n+1}\;:\;n\in\N\}, \leq)$\smallskip\\
\indent $\omega+\omega+1=ot(\{1-\frac1{n+1}\;:\;n\in\N\}\cup\{2-\frac1{n+1}\;:\;n\in\N\}\cup\{3\}, \leq)$\smallskip\\
\indent $\omega\cdot\omega=ot(\{m-\frac1n\;:\;n,m\in\N\},\leq)$\bigskip\\
{\large\color{acc}WŁASNOŚCI DZIAŁAŃ NA LICZBACH PORZĄDKOWYCH}\medskip\\
\indent - dodawanie i mnożenie są \emph{łączne}\smallskip\\
\indent - nie są przemienne - \emph{kolejność jest ważna}
$$\omega+1\neq1+\omega=\omega$$
\indent - mnożenie jest \emph{rozdzielne }względem dodawania\medskip\\
\podz{dygresyja}\medskip
\begin{center}\large
    {\color{def}NASTĘPNIKIEM }liczby porządkowej $\alpha$ nazywamy liczbę porządkową $\alpha\cup\{\alpha\}=\alpha+1=\beta$:\smallskip\\
    $Succ(\beta)\iff(\exists\;\alpha)\;On(\alpha)\;\land\;\beta=\alpha+1$\medskip\\
    {\color{def}LICZBĄ GRANICZNĄ} nazywamy liczbę porządkową $Lim(\beta)$, \\jeśli nie jest ona następnikiem innej liczby.\medskip\\
    \normalsize Najmniejszą liczbą graniczną jest 0, kolejną jest $\omega$, a wszytkie liczby \\naturalne są następnikami.
\end{center}\bigskip

{\large
$$Lim(\alpha)\iff\alpha=\bigcup\alpha$$}
\dowod
$\implies$\medskip\\
Wiem, że $Lim(\alpha)$, czyli
$$\neg\;(\exists\;\beta)\;\alpha=\beta\cup\{\beta\}.$$
Jeśli założymy, że \bigskip\\
$\impliedby$\medskip\\
Ponieważ $Tran(\alpha)$, to również $Tran(\bigcup\alpha)$. Załóżmy, nie wprost, że $Succ(\alpha)$, czyli
$$(\exists\;\beta)\;\alpha=\beta\cup\{\beta\}.$$
Wtedy
$$\bigcup\alpha=\bigcup(\beta\cup\{\beta\})=\beta,$$
ale wówczas
$$\beta\cup\{\beta\}=\beta,$$
czyli wówczas $\beta\in\beta\cup\{\beta\}=\beta$, co daje nam sprzeczność.

\subsection{INDUKCJA POZASKOŃCZONA}
\begin{center}\large
    Niech $\varphi(n)$ będzie formułą języka teorii mnogości taką, że\smallskip\\
    $(\forall\;\beta)(\forall\;\alpha<\beta)\;\varphi(\alpha)\implies\varphi(\beta)$\smallskip\\
    Wtedy $(\forall\;\alpha)\varphi(\alpha)$.\medskip\\
    Jest to {\color{def}TWIERDZENIE O INDUKCJI POZASKOŃCZONEJ}
\end{center}
\dowod
Przypuśćmy, nie wprost, że
$$(\exists\;\alpha)\neg\;\varphi(\alpha).$$
Wtedy zbiór
$$C=\{\gamma\in\alpha\cup\{\alpha\}\;:\;\varphi(\gamma)\}$$
jest niepustym zbiorem liczb porządkowych. Wtedy w $C$ istnieje element najmniejszy $\xi$. \\Jego minimalność oznacza, że
$$(\forall\;\varepsilon<\xi)\;\varphi(\varepsilon).$$
Z założenia, że
$$(\forall\;\alpha)(\forall\;\beta<\alpha)\;\varphi(\beta)\implies\varphi(\alpha)$$
wynika, że $\varphi(\xi)$, czyli mamy sprzeczność z $\xi\in C$.
\kondow
{\color{def}Struktura indukcji}:\medskip\\
\indent 1. krok bazowy - sprawdzamy dla najmniejszej możliwej liczby\smallskip\\
\indent 2. krok indukcyjny:\smallskip\\
\indent\indent - krok następnikowy\smallskip\\
\indent\indent - krok graniczny

\subsection{REKURSJA POZASKOŃCZONA}
Od twierdzenia o indukcji różni się swoją istotą - indukcja służy dowodzeniu, a re-\\kursja - tworzeniu konstrukcji.\bigskip
\begin{center}\large
    Niech $\varphi(x,y)$ będzie formułą języka teorii mnogości taką, że\smallskip\\
    $(\forall\;x)(\exists\;!y)\;\varphi(x,y).$\smallskip\\
    Wówczas dla każdej liczby porządkowej $\alpha$ istnieje funkcja $f$ taka, że\smallskip\\
    $dom(f)=\alpha$\smallskip\\
    i spełniony jest warunek\smallskip\\
    $(\forall\;\beta<\alpha)\;\varphi(f\obet\beta, f(\beta))\quad(\kawa)$
\end{center}
Tworzymy pozaskończony ciąg indeksowany liczbami porządkowymi, gdzie kolejny krok wynika z tego co juz mamy.\bigskip\\
\dowod
{\large\color{def}JEDYNOŚĆ}\medskip\\
Przypuśćmy, że dla pewnego $\alpha$ istnieją dwie różne funkcje $f_1,\;f_2$ o dziedzinie $\alpha$ spełnia-\\jące $(\kawa)$. Wtedy zbiór jest niepusty
$$\{\beta\in\alpha\;:\;f_1(\beta)\neq f_2(\beta)\}\neq\emptyset.$$
Niech $\beta_0$ będzie najmniejszym elementem tego zbioru. Wtedy dla $\varepsilon<\beta_0$ mamy
$$f_1(\varepsilon)=f_2(\varepsilon),$$
czyli $f_1\obet\beta_0=f_2\obet\beta_0$, czyli z $(\kawa)$ i $fnc(\varphi)$
$$f_1(\beta_0)=f_2(\beta_0),$$
co daje sprzeczność.\bigskip\\
{\large\color{def}ISTNIENIE}\medskip\\
Indukcja po $\alpha$\medskip\\
1. $\alpha=0$ OK\medskip\\
2. Krok indukcyjny\smallskip\\
Ustalmy $\alpha$ takie, że dla $\gamma<\alpha$ istnieje funkcja taka, że $dom(f)_\gamma=\gamma$ i spełnia $(\kawa)$.\smallskip\\
\indent krok następnikowy $\alpha=\beta+1$\smallskip\\
Wtedy istnieje $f_\beta$ jak powyżej. Wiemy, że istnieje dokładnie jedno $y$ takie, że zachodzi
$$\varphi(f_\beta, y).$$
Niech 
$$f_\alpha=f_\beta\cup\{\parl\beta, y\parr\}.$$
Wtedy $fnc(f_\alpha)$ oraz 
$$dom(f_\alpha)=dom(f_\beta)\cup\{\beta\}=\beta\cup\{\beta\}=\beta+1=\alpha.$$
Wystarczy pokazać, że $f_\alpha$ spełnia $(\kawa)$. Trzeba ustalić jakieś 
$$\eta<\alpha=\beta+1.$$
Więc jeśli $\eta<\beta$, to $f_\alpha\obet\eta=f_\beta\obet\eta$ oraz $f_\alpha(\eta)=f_\beta(\eta)$. Czyli spełnia z założenia indukcyj-\\nego.. \\
Jeśli $\eta=\beta$, to mamy $\varphi(f_\alpha\obet\beta, f_\alpha(\beta))$, bo $f_\alpha(\beta)=y$, co również jest prawdziwe.\medskip\\
\indent krok graniczny $Lim(\alpha)$\smallskip\\
Wiemy, że
$$Lim(\alpha)\iff \alpha=\bigcup \alpha.$$
